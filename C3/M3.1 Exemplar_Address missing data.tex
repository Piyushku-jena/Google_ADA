\documentclass[11pt]{article}

    \usepackage[breakable]{tcolorbox}
    \usepackage{parskip} % Stop auto-indenting (to mimic markdown behaviour)
    
    \usepackage{iftex}
    \ifPDFTeX
    	\usepackage[T1]{fontenc}
    	\usepackage{mathpazo}
    \else
    	\usepackage{fontspec}
    \fi

    % Basic figure setup, for now with no caption control since it's done
    % automatically by Pandoc (which extracts ![](path) syntax from Markdown).
    \usepackage{graphicx}
    % Maintain compatibility with old templates. Remove in nbconvert 6.0
    \let\Oldincludegraphics\includegraphics
    % Ensure that by default, figures have no caption (until we provide a
    % proper Figure object with a Caption API and a way to capture that
    % in the conversion process - todo).
    \usepackage{caption}
    \DeclareCaptionFormat{nocaption}{}
    \captionsetup{format=nocaption,aboveskip=0pt,belowskip=0pt}

    \usepackage[Export]{adjustbox} % Used to constrain images to a maximum size
    \adjustboxset{max size={0.9\linewidth}{0.9\paperheight}}
    \usepackage{float}
    \floatplacement{figure}{H} % forces figures to be placed at the correct location
    \usepackage{xcolor} % Allow colors to be defined
    \usepackage{enumerate} % Needed for markdown enumerations to work
    \usepackage{geometry} % Used to adjust the document margins
    \usepackage{amsmath} % Equations
    \usepackage{amssymb} % Equations
    \usepackage{textcomp} % defines textquotesingle
    % Hack from http://tex.stackexchange.com/a/47451/13684:
    \AtBeginDocument{%
        \def\PYZsq{\textquotesingle}% Upright quotes in Pygmentized code
    }
    \usepackage{upquote} % Upright quotes for verbatim code
    \usepackage{eurosym} % defines \euro
    \usepackage[mathletters]{ucs} % Extended unicode (utf-8) support
    \usepackage{fancyvrb} % verbatim replacement that allows latex
    \usepackage{grffile} % extends the file name processing of package graphics 
                         % to support a larger range
    \makeatletter % fix for grffile with XeLaTeX
    \def\Gread@@xetex#1{%
      \IfFileExists{"\Gin@base".bb}%
      {\Gread@eps{\Gin@base.bb}}%
      {\Gread@@xetex@aux#1}%
    }
    \makeatother

    % The hyperref package gives us a pdf with properly built
    % internal navigation ('pdf bookmarks' for the table of contents,
    % internal cross-reference links, web links for URLs, etc.)
    \usepackage{hyperref}
    % The default LaTeX title has an obnoxious amount of whitespace. By default,
    % titling removes some of it. It also provides customization options.
    \usepackage{titling}
    \usepackage{longtable} % longtable support required by pandoc >1.10
    \usepackage{booktabs}  % table support for pandoc > 1.12.2
    \usepackage[inline]{enumitem} % IRkernel/repr support (it uses the enumerate* environment)
    \usepackage[normalem]{ulem} % ulem is needed to support strikethroughs (\sout)
                                % normalem makes italics be italics, not underlines
    \usepackage{mathrsfs}
    

    
    % Colors for the hyperref package
    \definecolor{urlcolor}{rgb}{0,.145,.698}
    \definecolor{linkcolor}{rgb}{.71,0.21,0.01}
    \definecolor{citecolor}{rgb}{.12,.54,.11}

    % ANSI colors
    \definecolor{ansi-black}{HTML}{3E424D}
    \definecolor{ansi-black-intense}{HTML}{282C36}
    \definecolor{ansi-red}{HTML}{E75C58}
    \definecolor{ansi-red-intense}{HTML}{B22B31}
    \definecolor{ansi-green}{HTML}{00A250}
    \definecolor{ansi-green-intense}{HTML}{007427}
    \definecolor{ansi-yellow}{HTML}{DDB62B}
    \definecolor{ansi-yellow-intense}{HTML}{B27D12}
    \definecolor{ansi-blue}{HTML}{208FFB}
    \definecolor{ansi-blue-intense}{HTML}{0065CA}
    \definecolor{ansi-magenta}{HTML}{D160C4}
    \definecolor{ansi-magenta-intense}{HTML}{A03196}
    \definecolor{ansi-cyan}{HTML}{60C6C8}
    \definecolor{ansi-cyan-intense}{HTML}{258F8F}
    \definecolor{ansi-white}{HTML}{C5C1B4}
    \definecolor{ansi-white-intense}{HTML}{A1A6B2}
    \definecolor{ansi-default-inverse-fg}{HTML}{FFFFFF}
    \definecolor{ansi-default-inverse-bg}{HTML}{000000}

    % commands and environments needed by pandoc snippets
    % extracted from the output of `pandoc -s`
    \providecommand{\tightlist}{%
      \setlength{\itemsep}{0pt}\setlength{\parskip}{0pt}}
    \DefineVerbatimEnvironment{Highlighting}{Verbatim}{commandchars=\\\{\}}
    % Add ',fontsize=\small' for more characters per line
    \newenvironment{Shaded}{}{}
    \newcommand{\KeywordTok}[1]{\textcolor[rgb]{0.00,0.44,0.13}{\textbf{{#1}}}}
    \newcommand{\DataTypeTok}[1]{\textcolor[rgb]{0.56,0.13,0.00}{{#1}}}
    \newcommand{\DecValTok}[1]{\textcolor[rgb]{0.25,0.63,0.44}{{#1}}}
    \newcommand{\BaseNTok}[1]{\textcolor[rgb]{0.25,0.63,0.44}{{#1}}}
    \newcommand{\FloatTok}[1]{\textcolor[rgb]{0.25,0.63,0.44}{{#1}}}
    \newcommand{\CharTok}[1]{\textcolor[rgb]{0.25,0.44,0.63}{{#1}}}
    \newcommand{\StringTok}[1]{\textcolor[rgb]{0.25,0.44,0.63}{{#1}}}
    \newcommand{\CommentTok}[1]{\textcolor[rgb]{0.38,0.63,0.69}{\textit{{#1}}}}
    \newcommand{\OtherTok}[1]{\textcolor[rgb]{0.00,0.44,0.13}{{#1}}}
    \newcommand{\AlertTok}[1]{\textcolor[rgb]{1.00,0.00,0.00}{\textbf{{#1}}}}
    \newcommand{\FunctionTok}[1]{\textcolor[rgb]{0.02,0.16,0.49}{{#1}}}
    \newcommand{\RegionMarkerTok}[1]{{#1}}
    \newcommand{\ErrorTok}[1]{\textcolor[rgb]{1.00,0.00,0.00}{\textbf{{#1}}}}
    \newcommand{\NormalTok}[1]{{#1}}
    
    % Additional commands for more recent versions of Pandoc
    \newcommand{\ConstantTok}[1]{\textcolor[rgb]{0.53,0.00,0.00}{{#1}}}
    \newcommand{\SpecialCharTok}[1]{\textcolor[rgb]{0.25,0.44,0.63}{{#1}}}
    \newcommand{\VerbatimStringTok}[1]{\textcolor[rgb]{0.25,0.44,0.63}{{#1}}}
    \newcommand{\SpecialStringTok}[1]{\textcolor[rgb]{0.73,0.40,0.53}{{#1}}}
    \newcommand{\ImportTok}[1]{{#1}}
    \newcommand{\DocumentationTok}[1]{\textcolor[rgb]{0.73,0.13,0.13}{\textit{{#1}}}}
    \newcommand{\AnnotationTok}[1]{\textcolor[rgb]{0.38,0.63,0.69}{\textbf{\textit{{#1}}}}}
    \newcommand{\CommentVarTok}[1]{\textcolor[rgb]{0.38,0.63,0.69}{\textbf{\textit{{#1}}}}}
    \newcommand{\VariableTok}[1]{\textcolor[rgb]{0.10,0.09,0.49}{{#1}}}
    \newcommand{\ControlFlowTok}[1]{\textcolor[rgb]{0.00,0.44,0.13}{\textbf{{#1}}}}
    \newcommand{\OperatorTok}[1]{\textcolor[rgb]{0.40,0.40,0.40}{{#1}}}
    \newcommand{\BuiltInTok}[1]{{#1}}
    \newcommand{\ExtensionTok}[1]{{#1}}
    \newcommand{\PreprocessorTok}[1]{\textcolor[rgb]{0.74,0.48,0.00}{{#1}}}
    \newcommand{\AttributeTok}[1]{\textcolor[rgb]{0.49,0.56,0.16}{{#1}}}
    \newcommand{\InformationTok}[1]{\textcolor[rgb]{0.38,0.63,0.69}{\textbf{\textit{{#1}}}}}
    \newcommand{\WarningTok}[1]{\textcolor[rgb]{0.38,0.63,0.69}{\textbf{\textit{{#1}}}}}
    
    
    % Define a nice break command that doesn't care if a line doesn't already
    % exist.
    \def\br{\hspace*{\fill} \\* }
    % Math Jax compatibility definitions
    \def\gt{>}
    \def\lt{<}
    \let\Oldtex\TeX
    \let\Oldlatex\LaTeX
    \renewcommand{\TeX}{\textrm{\Oldtex}}
    \renewcommand{\LaTeX}{\textrm{\Oldlatex}}
    % Document parameters
    % Document title
    \title{Exemplar\_Address missing data}
    
    
    
    
    
% Pygments definitions
\makeatletter
\def\PY@reset{\let\PY@it=\relax \let\PY@bf=\relax%
    \let\PY@ul=\relax \let\PY@tc=\relax%
    \let\PY@bc=\relax \let\PY@ff=\relax}
\def\PY@tok#1{\csname PY@tok@#1\endcsname}
\def\PY@toks#1+{\ifx\relax#1\empty\else%
    \PY@tok{#1}\expandafter\PY@toks\fi}
\def\PY@do#1{\PY@bc{\PY@tc{\PY@ul{%
    \PY@it{\PY@bf{\PY@ff{#1}}}}}}}
\def\PY#1#2{\PY@reset\PY@toks#1+\relax+\PY@do{#2}}

\expandafter\def\csname PY@tok@w\endcsname{\def\PY@tc##1{\textcolor[rgb]{0.73,0.73,0.73}{##1}}}
\expandafter\def\csname PY@tok@c\endcsname{\let\PY@it=\textit\def\PY@tc##1{\textcolor[rgb]{0.25,0.50,0.50}{##1}}}
\expandafter\def\csname PY@tok@cp\endcsname{\def\PY@tc##1{\textcolor[rgb]{0.74,0.48,0.00}{##1}}}
\expandafter\def\csname PY@tok@k\endcsname{\let\PY@bf=\textbf\def\PY@tc##1{\textcolor[rgb]{0.00,0.50,0.00}{##1}}}
\expandafter\def\csname PY@tok@kp\endcsname{\def\PY@tc##1{\textcolor[rgb]{0.00,0.50,0.00}{##1}}}
\expandafter\def\csname PY@tok@kt\endcsname{\def\PY@tc##1{\textcolor[rgb]{0.69,0.00,0.25}{##1}}}
\expandafter\def\csname PY@tok@o\endcsname{\def\PY@tc##1{\textcolor[rgb]{0.40,0.40,0.40}{##1}}}
\expandafter\def\csname PY@tok@ow\endcsname{\let\PY@bf=\textbf\def\PY@tc##1{\textcolor[rgb]{0.67,0.13,1.00}{##1}}}
\expandafter\def\csname PY@tok@nb\endcsname{\def\PY@tc##1{\textcolor[rgb]{0.00,0.50,0.00}{##1}}}
\expandafter\def\csname PY@tok@nf\endcsname{\def\PY@tc##1{\textcolor[rgb]{0.00,0.00,1.00}{##1}}}
\expandafter\def\csname PY@tok@nc\endcsname{\let\PY@bf=\textbf\def\PY@tc##1{\textcolor[rgb]{0.00,0.00,1.00}{##1}}}
\expandafter\def\csname PY@tok@nn\endcsname{\let\PY@bf=\textbf\def\PY@tc##1{\textcolor[rgb]{0.00,0.00,1.00}{##1}}}
\expandafter\def\csname PY@tok@ne\endcsname{\let\PY@bf=\textbf\def\PY@tc##1{\textcolor[rgb]{0.82,0.25,0.23}{##1}}}
\expandafter\def\csname PY@tok@nv\endcsname{\def\PY@tc##1{\textcolor[rgb]{0.10,0.09,0.49}{##1}}}
\expandafter\def\csname PY@tok@no\endcsname{\def\PY@tc##1{\textcolor[rgb]{0.53,0.00,0.00}{##1}}}
\expandafter\def\csname PY@tok@nl\endcsname{\def\PY@tc##1{\textcolor[rgb]{0.63,0.63,0.00}{##1}}}
\expandafter\def\csname PY@tok@ni\endcsname{\let\PY@bf=\textbf\def\PY@tc##1{\textcolor[rgb]{0.60,0.60,0.60}{##1}}}
\expandafter\def\csname PY@tok@na\endcsname{\def\PY@tc##1{\textcolor[rgb]{0.49,0.56,0.16}{##1}}}
\expandafter\def\csname PY@tok@nt\endcsname{\let\PY@bf=\textbf\def\PY@tc##1{\textcolor[rgb]{0.00,0.50,0.00}{##1}}}
\expandafter\def\csname PY@tok@nd\endcsname{\def\PY@tc##1{\textcolor[rgb]{0.67,0.13,1.00}{##1}}}
\expandafter\def\csname PY@tok@s\endcsname{\def\PY@tc##1{\textcolor[rgb]{0.73,0.13,0.13}{##1}}}
\expandafter\def\csname PY@tok@sd\endcsname{\let\PY@it=\textit\def\PY@tc##1{\textcolor[rgb]{0.73,0.13,0.13}{##1}}}
\expandafter\def\csname PY@tok@si\endcsname{\let\PY@bf=\textbf\def\PY@tc##1{\textcolor[rgb]{0.73,0.40,0.53}{##1}}}
\expandafter\def\csname PY@tok@se\endcsname{\let\PY@bf=\textbf\def\PY@tc##1{\textcolor[rgb]{0.73,0.40,0.13}{##1}}}
\expandafter\def\csname PY@tok@sr\endcsname{\def\PY@tc##1{\textcolor[rgb]{0.73,0.40,0.53}{##1}}}
\expandafter\def\csname PY@tok@ss\endcsname{\def\PY@tc##1{\textcolor[rgb]{0.10,0.09,0.49}{##1}}}
\expandafter\def\csname PY@tok@sx\endcsname{\def\PY@tc##1{\textcolor[rgb]{0.00,0.50,0.00}{##1}}}
\expandafter\def\csname PY@tok@m\endcsname{\def\PY@tc##1{\textcolor[rgb]{0.40,0.40,0.40}{##1}}}
\expandafter\def\csname PY@tok@gh\endcsname{\let\PY@bf=\textbf\def\PY@tc##1{\textcolor[rgb]{0.00,0.00,0.50}{##1}}}
\expandafter\def\csname PY@tok@gu\endcsname{\let\PY@bf=\textbf\def\PY@tc##1{\textcolor[rgb]{0.50,0.00,0.50}{##1}}}
\expandafter\def\csname PY@tok@gd\endcsname{\def\PY@tc##1{\textcolor[rgb]{0.63,0.00,0.00}{##1}}}
\expandafter\def\csname PY@tok@gi\endcsname{\def\PY@tc##1{\textcolor[rgb]{0.00,0.63,0.00}{##1}}}
\expandafter\def\csname PY@tok@gr\endcsname{\def\PY@tc##1{\textcolor[rgb]{1.00,0.00,0.00}{##1}}}
\expandafter\def\csname PY@tok@ge\endcsname{\let\PY@it=\textit}
\expandafter\def\csname PY@tok@gs\endcsname{\let\PY@bf=\textbf}
\expandafter\def\csname PY@tok@gp\endcsname{\let\PY@bf=\textbf\def\PY@tc##1{\textcolor[rgb]{0.00,0.00,0.50}{##1}}}
\expandafter\def\csname PY@tok@go\endcsname{\def\PY@tc##1{\textcolor[rgb]{0.53,0.53,0.53}{##1}}}
\expandafter\def\csname PY@tok@gt\endcsname{\def\PY@tc##1{\textcolor[rgb]{0.00,0.27,0.87}{##1}}}
\expandafter\def\csname PY@tok@err\endcsname{\def\PY@bc##1{\setlength{\fboxsep}{0pt}\fcolorbox[rgb]{1.00,0.00,0.00}{1,1,1}{\strut ##1}}}
\expandafter\def\csname PY@tok@kc\endcsname{\let\PY@bf=\textbf\def\PY@tc##1{\textcolor[rgb]{0.00,0.50,0.00}{##1}}}
\expandafter\def\csname PY@tok@kd\endcsname{\let\PY@bf=\textbf\def\PY@tc##1{\textcolor[rgb]{0.00,0.50,0.00}{##1}}}
\expandafter\def\csname PY@tok@kn\endcsname{\let\PY@bf=\textbf\def\PY@tc##1{\textcolor[rgb]{0.00,0.50,0.00}{##1}}}
\expandafter\def\csname PY@tok@kr\endcsname{\let\PY@bf=\textbf\def\PY@tc##1{\textcolor[rgb]{0.00,0.50,0.00}{##1}}}
\expandafter\def\csname PY@tok@bp\endcsname{\def\PY@tc##1{\textcolor[rgb]{0.00,0.50,0.00}{##1}}}
\expandafter\def\csname PY@tok@fm\endcsname{\def\PY@tc##1{\textcolor[rgb]{0.00,0.00,1.00}{##1}}}
\expandafter\def\csname PY@tok@vc\endcsname{\def\PY@tc##1{\textcolor[rgb]{0.10,0.09,0.49}{##1}}}
\expandafter\def\csname PY@tok@vg\endcsname{\def\PY@tc##1{\textcolor[rgb]{0.10,0.09,0.49}{##1}}}
\expandafter\def\csname PY@tok@vi\endcsname{\def\PY@tc##1{\textcolor[rgb]{0.10,0.09,0.49}{##1}}}
\expandafter\def\csname PY@tok@vm\endcsname{\def\PY@tc##1{\textcolor[rgb]{0.10,0.09,0.49}{##1}}}
\expandafter\def\csname PY@tok@sa\endcsname{\def\PY@tc##1{\textcolor[rgb]{0.73,0.13,0.13}{##1}}}
\expandafter\def\csname PY@tok@sb\endcsname{\def\PY@tc##1{\textcolor[rgb]{0.73,0.13,0.13}{##1}}}
\expandafter\def\csname PY@tok@sc\endcsname{\def\PY@tc##1{\textcolor[rgb]{0.73,0.13,0.13}{##1}}}
\expandafter\def\csname PY@tok@dl\endcsname{\def\PY@tc##1{\textcolor[rgb]{0.73,0.13,0.13}{##1}}}
\expandafter\def\csname PY@tok@s2\endcsname{\def\PY@tc##1{\textcolor[rgb]{0.73,0.13,0.13}{##1}}}
\expandafter\def\csname PY@tok@sh\endcsname{\def\PY@tc##1{\textcolor[rgb]{0.73,0.13,0.13}{##1}}}
\expandafter\def\csname PY@tok@s1\endcsname{\def\PY@tc##1{\textcolor[rgb]{0.73,0.13,0.13}{##1}}}
\expandafter\def\csname PY@tok@mb\endcsname{\def\PY@tc##1{\textcolor[rgb]{0.40,0.40,0.40}{##1}}}
\expandafter\def\csname PY@tok@mf\endcsname{\def\PY@tc##1{\textcolor[rgb]{0.40,0.40,0.40}{##1}}}
\expandafter\def\csname PY@tok@mh\endcsname{\def\PY@tc##1{\textcolor[rgb]{0.40,0.40,0.40}{##1}}}
\expandafter\def\csname PY@tok@mi\endcsname{\def\PY@tc##1{\textcolor[rgb]{0.40,0.40,0.40}{##1}}}
\expandafter\def\csname PY@tok@il\endcsname{\def\PY@tc##1{\textcolor[rgb]{0.40,0.40,0.40}{##1}}}
\expandafter\def\csname PY@tok@mo\endcsname{\def\PY@tc##1{\textcolor[rgb]{0.40,0.40,0.40}{##1}}}
\expandafter\def\csname PY@tok@ch\endcsname{\let\PY@it=\textit\def\PY@tc##1{\textcolor[rgb]{0.25,0.50,0.50}{##1}}}
\expandafter\def\csname PY@tok@cm\endcsname{\let\PY@it=\textit\def\PY@tc##1{\textcolor[rgb]{0.25,0.50,0.50}{##1}}}
\expandafter\def\csname PY@tok@cpf\endcsname{\let\PY@it=\textit\def\PY@tc##1{\textcolor[rgb]{0.25,0.50,0.50}{##1}}}
\expandafter\def\csname PY@tok@c1\endcsname{\let\PY@it=\textit\def\PY@tc##1{\textcolor[rgb]{0.25,0.50,0.50}{##1}}}
\expandafter\def\csname PY@tok@cs\endcsname{\let\PY@it=\textit\def\PY@tc##1{\textcolor[rgb]{0.25,0.50,0.50}{##1}}}

\def\PYZbs{\char`\\}
\def\PYZus{\char`\_}
\def\PYZob{\char`\{}
\def\PYZcb{\char`\}}
\def\PYZca{\char`\^}
\def\PYZam{\char`\&}
\def\PYZlt{\char`\<}
\def\PYZgt{\char`\>}
\def\PYZsh{\char`\#}
\def\PYZpc{\char`\%}
\def\PYZdl{\char`\$}
\def\PYZhy{\char`\-}
\def\PYZsq{\char`\'}
\def\PYZdq{\char`\"}
\def\PYZti{\char`\~}
% for compatibility with earlier versions
\def\PYZat{@}
\def\PYZlb{[}
\def\PYZrb{]}
\makeatother


    % For linebreaks inside Verbatim environment from package fancyvrb. 
    \makeatletter
        \newbox\Wrappedcontinuationbox 
        \newbox\Wrappedvisiblespacebox 
        \newcommand*\Wrappedvisiblespace {\textcolor{red}{\textvisiblespace}} 
        \newcommand*\Wrappedcontinuationsymbol {\textcolor{red}{\llap{\tiny$\m@th\hookrightarrow$}}} 
        \newcommand*\Wrappedcontinuationindent {3ex } 
        \newcommand*\Wrappedafterbreak {\kern\Wrappedcontinuationindent\copy\Wrappedcontinuationbox} 
        % Take advantage of the already applied Pygments mark-up to insert 
        % potential linebreaks for TeX processing. 
        %        {, <, #, %, $, ' and ": go to next line. 
        %        _, }, ^, &, >, - and ~: stay at end of broken line. 
        % Use of \textquotesingle for straight quote. 
        \newcommand*\Wrappedbreaksatspecials {% 
            \def\PYGZus{\discretionary{\char`\_}{\Wrappedafterbreak}{\char`\_}}% 
            \def\PYGZob{\discretionary{}{\Wrappedafterbreak\char`\{}{\char`\{}}% 
            \def\PYGZcb{\discretionary{\char`\}}{\Wrappedafterbreak}{\char`\}}}% 
            \def\PYGZca{\discretionary{\char`\^}{\Wrappedafterbreak}{\char`\^}}% 
            \def\PYGZam{\discretionary{\char`\&}{\Wrappedafterbreak}{\char`\&}}% 
            \def\PYGZlt{\discretionary{}{\Wrappedafterbreak\char`\<}{\char`\<}}% 
            \def\PYGZgt{\discretionary{\char`\>}{\Wrappedafterbreak}{\char`\>}}% 
            \def\PYGZsh{\discretionary{}{\Wrappedafterbreak\char`\#}{\char`\#}}% 
            \def\PYGZpc{\discretionary{}{\Wrappedafterbreak\char`\%}{\char`\%}}% 
            \def\PYGZdl{\discretionary{}{\Wrappedafterbreak\char`\$}{\char`\$}}% 
            \def\PYGZhy{\discretionary{\char`\-}{\Wrappedafterbreak}{\char`\-}}% 
            \def\PYGZsq{\discretionary{}{\Wrappedafterbreak\textquotesingle}{\textquotesingle}}% 
            \def\PYGZdq{\discretionary{}{\Wrappedafterbreak\char`\"}{\char`\"}}% 
            \def\PYGZti{\discretionary{\char`\~}{\Wrappedafterbreak}{\char`\~}}% 
        } 
        % Some characters . , ; ? ! / are not pygmentized. 
        % This macro makes them "active" and they will insert potential linebreaks 
        \newcommand*\Wrappedbreaksatpunct {% 
            \lccode`\~`\.\lowercase{\def~}{\discretionary{\hbox{\char`\.}}{\Wrappedafterbreak}{\hbox{\char`\.}}}% 
            \lccode`\~`\,\lowercase{\def~}{\discretionary{\hbox{\char`\,}}{\Wrappedafterbreak}{\hbox{\char`\,}}}% 
            \lccode`\~`\;\lowercase{\def~}{\discretionary{\hbox{\char`\;}}{\Wrappedafterbreak}{\hbox{\char`\;}}}% 
            \lccode`\~`\:\lowercase{\def~}{\discretionary{\hbox{\char`\:}}{\Wrappedafterbreak}{\hbox{\char`\:}}}% 
            \lccode`\~`\?\lowercase{\def~}{\discretionary{\hbox{\char`\?}}{\Wrappedafterbreak}{\hbox{\char`\?}}}% 
            \lccode`\~`\!\lowercase{\def~}{\discretionary{\hbox{\char`\!}}{\Wrappedafterbreak}{\hbox{\char`\!}}}% 
            \lccode`\~`\/\lowercase{\def~}{\discretionary{\hbox{\char`\/}}{\Wrappedafterbreak}{\hbox{\char`\/}}}% 
            \catcode`\.\active
            \catcode`\,\active 
            \catcode`\;\active
            \catcode`\:\active
            \catcode`\?\active
            \catcode`\!\active
            \catcode`\/\active 
            \lccode`\~`\~ 	
        }
    \makeatother

    \let\OriginalVerbatim=\Verbatim
    \makeatletter
    \renewcommand{\Verbatim}[1][1]{%
        %\parskip\z@skip
        \sbox\Wrappedcontinuationbox {\Wrappedcontinuationsymbol}%
        \sbox\Wrappedvisiblespacebox {\FV@SetupFont\Wrappedvisiblespace}%
        \def\FancyVerbFormatLine ##1{\hsize\linewidth
            \vtop{\raggedright\hyphenpenalty\z@\exhyphenpenalty\z@
                \doublehyphendemerits\z@\finalhyphendemerits\z@
                \strut ##1\strut}%
        }%
        % If the linebreak is at a space, the latter will be displayed as visible
        % space at end of first line, and a continuation symbol starts next line.
        % Stretch/shrink are however usually zero for typewriter font.
        \def\FV@Space {%
            \nobreak\hskip\z@ plus\fontdimen3\font minus\fontdimen4\font
            \discretionary{\copy\Wrappedvisiblespacebox}{\Wrappedafterbreak}
            {\kern\fontdimen2\font}%
        }%
        
        % Allow breaks at special characters using \PYG... macros.
        \Wrappedbreaksatspecials
        % Breaks at punctuation characters . , ; ? ! and / need catcode=\active 	
        \OriginalVerbatim[#1,codes*=\Wrappedbreaksatpunct]%
    }
    \makeatother

    % Exact colors from NB
    \definecolor{incolor}{HTML}{303F9F}
    \definecolor{outcolor}{HTML}{D84315}
    \definecolor{cellborder}{HTML}{CFCFCF}
    \definecolor{cellbackground}{HTML}{F7F7F7}
    
    % prompt
    \makeatletter
    \newcommand{\boxspacing}{\kern\kvtcb@left@rule\kern\kvtcb@boxsep}
    \makeatother
    \newcommand{\prompt}[4]{
        \ttfamily\llap{{\color{#2}[#3]:\hspace{3pt}#4}}\vspace{-\baselineskip}
    }
    

    
    % Prevent overflowing lines due to hard-to-break entities
    \sloppy 
    % Setup hyperref package
    \hypersetup{
      breaklinks=true,  % so long urls are correctly broken across lines
      colorlinks=true,
      urlcolor=urlcolor,
      linkcolor=linkcolor,
      citecolor=citecolor,
      }
    % Slightly bigger margins than the latex defaults
    
    \geometry{verbose,tmargin=1in,bmargin=1in,lmargin=1in,rmargin=1in}
    
    

\begin{document}
    
    \maketitle
    
    

    
    \hypertarget{exemplar-address-missing-data}{%
\section{Exemplar: Address missing
data}\label{exemplar-address-missing-data}}

    \hypertarget{introduction}{%
\subsection{Introduction}\label{introduction}}

The datasets that data professionals use to solve problems typically
contain missing values, which must be dealt with in order to achieve
clean, useful data. This is particularly crucial in exploratory data
analysis (EDA). In this activity, you will learn how to address missing
data.

You are a financial data consultant, and an investor has tasked your
team with identifying new business opportunities. To help them decide
which future companies to invest in, you will provide a list of current
businesses valued at more than \$1 billion. These are sometimes referred
to as ``unicorns.'' Your client will use this information to learn about
profitable businesses in general.

The investor has asked you to provide them with the following data: -
Companies in the \texttt{hardware} industry based in \texttt{Beijing},
\texttt{San\ Francisco}, and \texttt{London} - Companies in the
\texttt{artificial\ intelligence} industry based in \texttt{London} - A
list of the top 20 countries sorted by sum of company valuations in each
country, excluding \texttt{United\ States}, \texttt{China},
\texttt{India}, and \texttt{United\ Kingdom} - A global valuation map of
all countries except \texttt{United\ States}, \texttt{China},
\texttt{India}, and \texttt{United\ Kingdom}

Your dataset includes a list of businesses and data points, such as the
year they were founded; their industry; and their city, country, and
continent.

    \hypertarget{step-1-imports}{%
\subsection{Step 1: Imports}\label{step-1-imports}}

    \hypertarget{import-libraries}{%
\subsubsection{Import libraries}\label{import-libraries}}

    Import the following relevant Python libraries: * \texttt{numpy} *
\texttt{pandas} * \texttt{matplotlib.pyplot} * \texttt{plotly.express} *
\texttt{seaborn}

    \begin{tcolorbox}[breakable, size=fbox, boxrule=1pt, pad at break*=1mm,colback=cellbackground, colframe=cellborder]
\prompt{In}{incolor}{1}{\boxspacing}
\begin{Verbatim}[commandchars=\\\{\}]
\PY{c+c1}{\PYZsh{} Import libraries and modules}

\PY{c+c1}{\PYZsh{}\PYZsh{}\PYZsh{} YOUR CODE HERE \PYZsh{}\PYZsh{}\PYZsh{}}

\PY{k+kn}{import} \PY{n+nn}{numpy} \PY{k}{as} \PY{n+nn}{np}
\PY{k+kn}{import} \PY{n+nn}{pandas} \PY{k}{as} \PY{n+nn}{pd}
\PY{k+kn}{import} \PY{n+nn}{matplotlib}\PY{n+nn}{.}\PY{n+nn}{pyplot} \PY{k}{as} \PY{n+nn}{plt}
\PY{k+kn}{import} \PY{n+nn}{plotly}\PY{n+nn}{.}\PY{n+nn}{express} \PY{k}{as} \PY{n+nn}{px}
\PY{k+kn}{import} \PY{n+nn}{seaborn} \PY{k}{as} \PY{n+nn}{sns}
\end{Verbatim}
\end{tcolorbox}

    \hypertarget{load-the-dataset}{%
\subsubsection{Load the dataset}\label{load-the-dataset}}

    The dataset is currently in CSV format and in a file named
\texttt{Unicorn\_Companies.csv}. As shown in this cell, the dataset has
been automatically loaded in for you. You do not need to download the
.csv file, or provide more code, in order to access the dataset and
proceed with this lab. Please continue with this activity by completing
the following instructions.

    \begin{tcolorbox}[breakable, size=fbox, boxrule=1pt, pad at break*=1mm,colback=cellbackground, colframe=cellborder]
\prompt{In}{incolor}{2}{\boxspacing}
\begin{Verbatim}[commandchars=\\\{\}]
\PY{c+c1}{\PYZsh{} Read the data into a dataframe}

\PY{c+c1}{\PYZsh{}\PYZsh{}\PYZsh{} YOUR CODE HERE \PYZsh{}\PYZsh{}\PYZsh{}}

\PY{n}{df\PYZus{}companies} \PY{o}{=} \PY{n}{pd}\PY{o}{.}\PY{n}{read\PYZus{}csv}\PY{p}{(}\PY{l+s+s1}{\PYZsq{}}\PY{l+s+s1}{Unicorn\PYZus{}Companies.csv}\PY{l+s+s1}{\PYZsq{}}\PY{p}{)}
\end{Verbatim}
\end{tcolorbox}

    \hypertarget{step-2-data-exploration}{%
\subsection{Step 2: Data exploration}\label{step-2-data-exploration}}

Explore the dataset and answer questions that will guide your management
of missing values.

    \hypertarget{display-top-rows}{%
\subsubsection{Display top rows}\label{display-top-rows}}

    Display the first 10 rows of the data to understand how the dataset is
structured.

    \begin{tcolorbox}[breakable, size=fbox, boxrule=1pt, pad at break*=1mm,colback=cellbackground, colframe=cellborder]
\prompt{In}{incolor}{3}{\boxspacing}
\begin{Verbatim}[commandchars=\\\{\}]
\PY{c+c1}{\PYZsh{} Display the first 10 rows of the data.}

\PY{c+c1}{\PYZsh{}\PYZsh{}\PYZsh{} YOUR CODE HERE \PYZsh{}\PYZsh{}\PYZsh{}}

\PY{n}{df\PYZus{}companies}\PY{o}{.}\PY{n}{head}\PY{p}{(}\PY{l+m+mi}{10}\PY{p}{)}
\end{Verbatim}
\end{tcolorbox}

            \begin{tcolorbox}[breakable, size=fbox, boxrule=.5pt, pad at break*=1mm, opacityfill=0]
\prompt{Out}{outcolor}{3}{\boxspacing}
\begin{Verbatim}[commandchars=\\\{\}]
        Company Valuation Date Joined                             Industry  \textbackslash{}
0     Bytedance     \$180B      4/7/17              Artificial intelligence
1        SpaceX     \$100B     12/1/12                                Other
2         SHEIN     \$100B      7/3/18      E-commerce \& direct-to-consumer
3        Stripe      \$95B     1/23/14                              Fintech
4        Klarna      \$46B    12/12/11                              Fintech
5         Canva      \$40B      1/8/18         Internet software \& services
6  Checkout.com      \$40B      5/2/19                              Fintech
7     Instacart      \$39B    12/30/14  Supply chain, logistics, \& delivery
8     JUUL Labs      \$38B    12/20/17                    Consumer \& retail
9    Databricks      \$38B      2/5/19          Data management \& analytics

            City  Country/Region      Continent  Year Founded Funding  \textbackslash{}
0        Beijing           China           Asia          2012     \$8B
1      Hawthorne   United States  North America          2002     \$7B
2       Shenzhen           China           Asia          2008     \$2B
3  San Francisco   United States  North America          2010     \$2B
4      Stockholm          Sweden         Europe          2005     \$4B
5    Surry Hills       Australia        Oceania          2012   \$572M
6         London  United Kingdom         Europe          2012     \$2B
7  San Francisco   United States  North America          2012     \$3B
8  San Francisco   United States  North America          2015    \$14B
9  San Francisco   United States  North America          2013     \$3B

                                    Select Investors
0  Sequoia Capital China, SIG Asia Investments, S{\ldots}
1  Founders Fund, Draper Fisher Jurvetson, Rothen{\ldots}
2  Tiger Global Management, Sequoia Capital China{\ldots}
3        Khosla Ventures, LowercaseCapital, capitalG
4  Institutional Venture Partners, Sequoia Capita{\ldots}
5  Sequoia Capital China, Blackbird Ventures, Mat{\ldots}
6  Tiger Global Management, Insight Partners, DST{\ldots}
7  Khosla Ventures, Kleiner Perkins Caufield \& By{\ldots}
8                            Tiger Global Management
9  Andreessen Horowitz, New Enterprise Associates{\ldots}
\end{Verbatim}
\end{tcolorbox}
        
    Hint 1

Refer to the materials about exploratory data analysis in Python.

    Hint 2

There is a function in the \texttt{pandas} library that allows you to
get a specific number of rows from the top of a DataFrame.

    Hint 3

Call the \texttt{head(10)} method on the dataframe.

    \hypertarget{statistical-properties-of-the-dataset}{%
\subsubsection{Statistical properties of the
dataset}\label{statistical-properties-of-the-dataset}}

    Use methods and attributes of the dataframe to get information and
descriptive statistics for the data, including its range, data types,
mean values, and shape.

    \begin{tcolorbox}[breakable, size=fbox, boxrule=1pt, pad at break*=1mm,colback=cellbackground, colframe=cellborder]
\prompt{In}{incolor}{4}{\boxspacing}
\begin{Verbatim}[commandchars=\\\{\}]
\PY{c+c1}{\PYZsh{} Get the shape of the dataset.}

\PY{c+c1}{\PYZsh{}\PYZsh{}\PYZsh{} YOUR CODE HERE \PYZsh{}\PYZsh{}\PYZsh{}}

\PY{n}{df\PYZus{}companies}\PY{o}{.}\PY{n}{shape}
\end{Verbatim}
\end{tcolorbox}

            \begin{tcolorbox}[breakable, size=fbox, boxrule=.5pt, pad at break*=1mm, opacityfill=0]
\prompt{Out}{outcolor}{4}{\boxspacing}
\begin{Verbatim}[commandchars=\\\{\}]
(1074, 10)
\end{Verbatim}
\end{tcolorbox}
        
    Hint 1

Refer to the material about exploratory data analysis in Python.

    Hint 2

Call the \texttt{shape} attribute of the dataframe.

    \textbf{Question: What is the shape of the dataset?} - (1074, 10)

    \begin{tcolorbox}[breakable, size=fbox, boxrule=1pt, pad at break*=1mm,colback=cellbackground, colframe=cellborder]
\prompt{In}{incolor}{5}{\boxspacing}
\begin{Verbatim}[commandchars=\\\{\}]
\PY{c+c1}{\PYZsh{} Get the data types and number of non\PYZhy{}null values in the dataset.}

\PY{c+c1}{\PYZsh{}\PYZsh{}\PYZsh{} YOUR CODE HERE \PYZsh{}\PYZsh{}\PYZsh{}}

\PY{n}{df\PYZus{}companies}\PY{o}{.}\PY{n}{info}\PY{p}{(}\PY{p}{)}
\end{Verbatim}
\end{tcolorbox}

    \begin{Verbatim}[commandchars=\\\{\}]
<class 'pandas.core.frame.DataFrame'>
RangeIndex: 1074 entries, 0 to 1073
Data columns (total 10 columns):
 \#   Column            Non-Null Count  Dtype
---  ------            --------------  -----
 0   Company           1074 non-null   object
 1   Valuation         1074 non-null   object
 2   Date Joined       1074 non-null   object
 3   Industry          1074 non-null   object
 4   City              1058 non-null   object
 5   Country/Region    1074 non-null   object
 6   Continent         1074 non-null   object
 7   Year Founded      1074 non-null   int64
 8   Funding           1074 non-null   object
 9   Select Investors  1073 non-null   object
dtypes: int64(1), object(9)
memory usage: 84.0+ KB
    \end{Verbatim}

    Hint 1

Refer to the material about exploratory data analysis in Python.

    Hint 2

Use the \texttt{info()} method.

    \textbf{Question: What are the data types of various columns?} - Except
for the \texttt{Year\ Funded} column, which is \texttt{int64}, the data
type for all other columns is \texttt{object}. You can get this
information using the \texttt{info()} method.

    \textbf{Question: How many columns contain null values?} - Columns
\texttt{City} and \texttt{Select\ Investors} contain fewer non-null
values than the total number of rows in the dataset, which indicates
that they are missing values. You can get this information using the
\texttt{info()} method.

    \begin{tcolorbox}[breakable, size=fbox, boxrule=1pt, pad at break*=1mm,colback=cellbackground, colframe=cellborder]
\prompt{In}{incolor}{6}{\boxspacing}
\begin{Verbatim}[commandchars=\\\{\}]
\PY{c+c1}{\PYZsh{} Get descriptive statistics such as mean, standard deviation, and range of the numerical columns in the dataset.}

\PY{c+c1}{\PYZsh{}\PYZsh{}\PYZsh{} YOUR CODE HERE \PYZsh{}\PYZsh{}\PYZsh{}}

\PY{n}{df\PYZus{}companies}\PY{o}{.}\PY{n}{describe}\PY{p}{(}\PY{p}{)}
\end{Verbatim}
\end{tcolorbox}

            \begin{tcolorbox}[breakable, size=fbox, boxrule=.5pt, pad at break*=1mm, opacityfill=0]
\prompt{Out}{outcolor}{6}{\boxspacing}
\begin{Verbatim}[commandchars=\\\{\}]
       Year Founded
count   1074.000000
mean    2012.895717
std        5.698573
min     1919.000000
25\%     2011.000000
50\%     2014.000000
75\%     2016.000000
max     2021.000000
\end{Verbatim}
\end{tcolorbox}
        
    Hint 1

Refer to the material about exploratory data analysis in Python.

    Hint 2

There is a function in the \texttt{pandas} library that allows you to
find descriptive statistics for the numeric columns in a DataFrame.

    Hint 3

Call the \texttt{describe()} method on the dataframe.

    \textbf{Question: In what year was the oldest company founded?} - The
oldest company in the list was founded in 1919. This is the minimum
value in \texttt{Year\ Funded}. You can get this information using the
\texttt{describe()} method.

    \hypertarget{data-preprocessing}{%
\subsubsection{Data Preprocessing}\label{data-preprocessing}}

    In order to answer the investor's questions, some data preprocessing
steps are required. The first step is to add a new column to the
dataframe containing just the year each company became a unicorn
company. Call this new column \texttt{Year\ Joined}.

    \begin{tcolorbox}[breakable, size=fbox, boxrule=1pt, pad at break*=1mm,colback=cellbackground, colframe=cellborder]
\prompt{In}{incolor}{7}{\boxspacing}
\begin{Verbatim}[commandchars=\\\{\}]
\PY{c+c1}{\PYZsh{} Create a new column `Year Joined` }

\PY{c+c1}{\PYZsh{}\PYZsh{}\PYZsh{} YOUR CODE HERE \PYZsh{}\PYZsh{}\PYZsh{}}

\PY{n}{df\PYZus{}companies}\PY{p}{[}\PY{l+s+s1}{\PYZsq{}}\PY{l+s+s1}{Year\PYZus{}Joined}\PY{l+s+s1}{\PYZsq{}}\PY{p}{]} \PY{o}{=} \PY{n}{pd}\PY{o}{.}\PY{n}{to\PYZus{}datetime}\PY{p}{(}\PY{n}{df\PYZus{}companies}\PY{p}{[}\PY{l+s+s1}{\PYZsq{}}\PY{l+s+s1}{Date Joined}\PY{l+s+s1}{\PYZsq{}}\PY{p}{]}\PY{p}{)}\PY{o}{.}\PY{n}{dt}\PY{o}{.}\PY{n}{year}
\end{Verbatim}
\end{tcolorbox}

    For each country, you want to calculate the sum of all valuations of
companies from that country. However, in order to do this, you'll need
to first prepare the data. Currently, the data in the \texttt{Valuation}
column is a string that starts with a \texttt{\$} and ends with a
\texttt{B}. Because this column is not in a numeric datatype, pandas
cannot perform mathematical operations on its values. The data in this
column needs to be converted to a numeric datatype.

In this step, define a function called \texttt{str\_to\_num()} that
accepts as an argument:

\begin{itemize}
\tightlist
\item
  \texttt{x}: a string in the format of the values contained in the
  \texttt{Valuation} column
\end{itemize}

And returns:

\begin{itemize}
\tightlist
\item
  \texttt{x}: an \texttt{int} of the number represented by the input
  string
\end{itemize}

\begin{verbatim}
Example:

 [IN]:  str_to_num('$4B')
[OUT]:  4
\end{verbatim}

To do this, use the string
\href{https://docs.python.org/3/library/stdtypes.html\#str.strip}{\texttt{strip()}}
method. This method is applied to a string. Its argument is a string
that contains all the characters that you want to remove from the
beginning and end of a given string---in any order. The specified
characters will be removed until a valid character is encountered. This
process is applied moving forward from the beginning of the string and
also moving in reverse from the end of the string, thus removing
unwanted beginning and trailing characters.

\begin{verbatim}
Example:

 [IN]:  my_string = '#....... Section 3.2.1 Issue #32 .......'
        my_string = my_string.strip('.#! ')
        print(my_string)

[OUT]:  'Section 3.2.1 Issue #32'
\end{verbatim}

Note that you must reassign the result back to a variable or else the
change will not be permanent.

    \begin{tcolorbox}[breakable, size=fbox, boxrule=1pt, pad at break*=1mm,colback=cellbackground, colframe=cellborder]
\prompt{In}{incolor}{8}{\boxspacing}
\begin{Verbatim}[commandchars=\\\{\}]
\PY{c+c1}{\PYZsh{} Define the `str\PYZus{}to\PYZus{}num()` function}

\PY{c+c1}{\PYZsh{}\PYZsh{}\PYZsh{} YOUR CODE HERE \PYZsh{}\PYZsh{}\PYZsh{}}
\PY{k}{def} \PY{n+nf}{str\PYZus{}to\PYZus{}num}\PY{p}{(}\PY{n}{x}\PY{p}{)}\PY{p}{:}
    \PY{n}{x} \PY{o}{=} \PY{n}{x}\PY{o}{.}\PY{n}{strip}\PY{p}{(}\PY{l+s+s1}{\PYZsq{}}\PY{l+s+s1}{\PYZdl{}B}\PY{l+s+s1}{\PYZsq{}}\PY{p}{)}
    \PY{n}{x} \PY{o}{=} \PY{n+nb}{int}\PY{p}{(}\PY{n}{x}\PY{p}{)}

    \PY{k}{return} \PY{n}{x}
\end{Verbatim}
\end{tcolorbox}

    Hint 1

The unwanted characters in the values contained in the
\texttt{Valuation} column are
\texttt{\textquotesingle{}\$\textquotesingle{}} and
\texttt{\textquotesingle{}B\textquotesingle{}}.

    Hint 2

Pass a string of the unwanted values as an argument to the
\texttt{strip()} string method.

    Hint 3

The final step before returning \texttt{x} should be converting it to an
integer.

    Now, use this function to create a new column called
\texttt{valuation\_num} that represents the \texttt{Valuation} column as
an integer value. To do this, use the series method
\href{https://pandas.pydata.org/docs/reference/api/pandas.Series.apply.html}{\texttt{apply()}}
to apply the \texttt{str\_to\_num()} function to the \texttt{Valuation}
column.

\texttt{apply()} is a method that can be used on a \texttt{DataFrame} or
\texttt{Series} object. In this case, you're using it on the
\texttt{Valuation} series. The method accepts a function as an argument
and applies that function to each value in the series.

\begin{verbatim}
Example:

 [IN]: def square(x):
           return x ** 2

       my_series = pd.Series([0, 1, 2, 3])
       my_series

[OUT]: 0    0
       1    1
       2    2
       3    3
       dtype: int64

 [IN]: my_series = my_series.apply(square)
       my_series

[OUT]: 0    0
       1    1
       2    4
       3    9
       dtype: int64
\end{verbatim}

Notice that the function passed as an argument to the \texttt{apply()}
method does not have parentheses. It's just the function name.

    \begin{tcolorbox}[breakable, size=fbox, boxrule=1pt, pad at break*=1mm,colback=cellbackground, colframe=cellborder]
\prompt{In}{incolor}{9}{\boxspacing}
\begin{Verbatim}[commandchars=\\\{\}]
\PY{c+c1}{\PYZsh{} Apply the `str\PYZus{}to\PYZus{}num()` function to the `Valuation` column}
\PY{c+c1}{\PYZsh{} and assign the result back to a new column called `valuation\PYZus{}num`}

\PY{c+c1}{\PYZsh{}\PYZsh{}\PYZsh{} YOUR CODE HERE \PYZsh{}\PYZsh{}\PYZsh{}}
\PY{n}{df\PYZus{}companies}\PY{p}{[}\PY{l+s+s1}{\PYZsq{}}\PY{l+s+s1}{valuation\PYZus{}num}\PY{l+s+s1}{\PYZsq{}}\PY{p}{]} \PY{o}{=} \PY{n}{df\PYZus{}companies}\PY{p}{[}\PY{l+s+s1}{\PYZsq{}}\PY{l+s+s1}{Valuation}\PY{l+s+s1}{\PYZsq{}}\PY{p}{]}\PY{o}{.}\PY{n}{apply}\PY{p}{(}\PY{n}{str\PYZus{}to\PYZus{}num}\PY{p}{)}
\PY{n}{df\PYZus{}companies}\PY{p}{[}\PY{p}{[}\PY{l+s+s1}{\PYZsq{}}\PY{l+s+s1}{Valuation}\PY{l+s+s1}{\PYZsq{}}\PY{p}{,} \PY{l+s+s1}{\PYZsq{}}\PY{l+s+s1}{valuation\PYZus{}num}\PY{l+s+s1}{\PYZsq{}}\PY{p}{]}\PY{p}{]}\PY{o}{.}\PY{n}{head}\PY{p}{(}\PY{p}{)}
\end{Verbatim}
\end{tcolorbox}

            \begin{tcolorbox}[breakable, size=fbox, boxrule=.5pt, pad at break*=1mm, opacityfill=0]
\prompt{Out}{outcolor}{9}{\boxspacing}
\begin{Verbatim}[commandchars=\\\{\}]
  Valuation  valuation\_num
0     \$180B            180
1     \$100B            100
2     \$100B            100
3      \$95B             95
4      \$46B             46
\end{Verbatim}
\end{tcolorbox}
        
    \hypertarget{find-missing-values}{%
\subsubsection{Find missing values}\label{find-missing-values}}

    The unicorn companies dataset is fairly clean, with few missing values.

    \begin{tcolorbox}[breakable, size=fbox, boxrule=1pt, pad at break*=1mm,colback=cellbackground, colframe=cellborder]
\prompt{In}{incolor}{10}{\boxspacing}
\begin{Verbatim}[commandchars=\\\{\}]
\PY{c+c1}{\PYZsh{} Find the number of missing values in each column in this dataset.}

\PY{c+c1}{\PYZsh{}\PYZsh{}\PYZsh{} YOUR CODE HERE \PYZsh{}\PYZsh{}\PYZsh{}}

\PY{n}{df\PYZus{}companies}\PY{o}{.}\PY{n}{isna}\PY{p}{(}\PY{p}{)}\PY{o}{.}\PY{n}{sum}\PY{p}{(}\PY{p}{)}
\end{Verbatim}
\end{tcolorbox}

            \begin{tcolorbox}[breakable, size=fbox, boxrule=.5pt, pad at break*=1mm, opacityfill=0]
\prompt{Out}{outcolor}{10}{\boxspacing}
\begin{Verbatim}[commandchars=\\\{\}]
Company              0
Valuation            0
Date Joined          0
Industry             0
City                16
Country/Region       0
Continent            0
Year Founded         0
Funding              0
Select Investors     1
Year\_Joined          0
valuation\_num        0
dtype: int64
\end{Verbatim}
\end{tcolorbox}
        
    Hint 1

The \texttt{isna()} \texttt{DataFrame} method will return a dataframe of
Boolean values in the same shape as your original dataframe. Values are
\texttt{True} if the data is missing and \texttt{False} if it is not
missing.

    Hint 2

You'll need to convert Boolean values into numerical values. Remember
that \texttt{True} values are considered \texttt{1} and \texttt{False}
values are considered \texttt{0}.

    Hint 3

After applying the \texttt{isna()} method to the \texttt{df\_companies}
dataframe, apply the \texttt{sum()} method to the results to return a
pandas \texttt{Series} object with each column name and the number of
\texttt{NaN} values it contains.

    \textbf{Question: How many missing values are in each column in the
dataset?} - There is a single missing value in the
\texttt{Select\ Investors} column and 16 missing cities. There are no
missing values in other columns.

    \hypertarget{review-rows-with-missing-values}{%
\subsubsection{Review rows with missing
values}\label{review-rows-with-missing-values}}

    Before dealing with missing values, it's important to understand the
nature of the missing value that is being filled. Display all rows with
missing values from \texttt{df\_companies}. To do this, perform the
following three steps:

\begin{enumerate}
\def\labelenumi{\arabic{enumi}.}
\tightlist
\item
  Apply the \texttt{isna()} method to the \texttt{df\_companies}
  dataframe as you did in the last step. Remember, this results in a
  dataframe of the same shape as \texttt{df\_companies} where each value
  is \texttt{True} if its contents are \texttt{NaN} and a \texttt{False}
  if its contents are not \texttt{NaN}. Assign the results to a variable
  called \texttt{mask}.
\end{enumerate}

    \begin{tcolorbox}[breakable, size=fbox, boxrule=1pt, pad at break*=1mm,colback=cellbackground, colframe=cellborder]
\prompt{In}{incolor}{11}{\boxspacing}
\begin{Verbatim}[commandchars=\\\{\}]
\PY{c+c1}{\PYZsh{} 1. Apply the `isna()` method to the `df\PYZus{}companies` dataframe and assign back to `mask`}
\PY{n}{mask} \PY{o}{=} \PY{n}{df\PYZus{}companies}\PY{o}{.}\PY{n}{isna}\PY{p}{(}\PY{p}{)}
\PY{n}{mask}\PY{o}{.}\PY{n}{tail}\PY{p}{(}\PY{p}{)}
\end{Verbatim}
\end{tcolorbox}

            \begin{tcolorbox}[breakable, size=fbox, boxrule=.5pt, pad at break*=1mm, opacityfill=0]
\prompt{Out}{outcolor}{11}{\boxspacing}
\begin{Verbatim}[commandchars=\\\{\}]
      Company  Valuation  Date Joined  Industry   City  Country/Region  \textbackslash{}
1069    False      False        False     False  False           False
1070    False      False        False     False  False           False
1071    False      False        False     False  False           False
1072    False      False        False     False  False           False
1073    False      False        False     False  False           False

      Continent  Year Founded  Funding  Select Investors  Year\_Joined  \textbackslash{}
1069      False         False    False             False        False
1070      False         False    False             False        False
1071      False         False    False             False        False
1072      False         False    False             False        False
1073      False         False    False             False        False

      valuation\_num
1069          False
1070          False
1071          False
1072          False
1073          False
\end{Verbatim}
\end{tcolorbox}
        
    You're not done yet. You still need to go from this dataframe of Boolean
values to a dataframe of just the rows of \texttt{df\_companies} that
contain at least one \texttt{NaN} value. This means that you need a way
to find the indices of the rows of the Boolean dataframe that contain at
least one \texttt{True} value, then extract those indices from
\texttt{df\_companies}.

You can do this using the
\href{https://pandas.pydata.org/docs/reference/api/pandas.DataFrame.any.html}{\texttt{any()}}
method for \texttt{DataFrame} objects. This method returns a Boolean
\texttt{Series} indicating whether any value is \texttt{True} over a
specified axis.

\begin{verbatim}
Example:

df =     
        A      B    C
    0   0      a    10
    1   False  0    1
    2   NaN    NaN  NaN


 [IN]: df.any(axis=0)

[OUT]: A    False
       B     True
       C     True
       dtype: bool
       
 [IN]: df.any(axis=1)
 
[OUT]: 0     True
       1     True
       2    False
       dtype: bool
\end{verbatim}

Note that \texttt{0}, \texttt{False}, and \texttt{NaN} are considered
\texttt{False} and anything else is considered \texttt{True}.

\begin{enumerate}
\def\labelenumi{\arabic{enumi}.}
\setcounter{enumi}{1}
\tightlist
\item
  Apply the \texttt{any()} method to the Boolean dataframe you created
  to make a Boolean series where each element in the series represents
  \texttt{True} if a row of the dataframe contains any \texttt{True}
  values and \texttt{False} if any row in the dataframe contains any
  \texttt{False} values. Assign the results back to \texttt{mask}.
\end{enumerate}

    \begin{tcolorbox}[breakable, size=fbox, boxrule=1pt, pad at break*=1mm,colback=cellbackground, colframe=cellborder]
\prompt{In}{incolor}{12}{\boxspacing}
\begin{Verbatim}[commandchars=\\\{\}]
\PY{c+c1}{\PYZsh{} 2. Apply the `any()` method to `mask` and assign the results back to `mask`}

\PY{c+c1}{\PYZsh{}\PYZsh{}\PYZsh{} YOUR CODE HERE \PYZsh{}\PYZsh{}\PYZsh{}}

\PY{n}{mask} \PY{o}{=} \PY{n}{mask}\PY{o}{.}\PY{n}{any}\PY{p}{(}\PY{n}{axis}\PY{o}{=}\PY{l+m+mi}{1}\PY{p}{)}
\PY{n}{mask}\PY{o}{.}\PY{n}{head}\PY{p}{(}\PY{p}{)}
\end{Verbatim}
\end{tcolorbox}

            \begin{tcolorbox}[breakable, size=fbox, boxrule=.5pt, pad at break*=1mm, opacityfill=0]
\prompt{Out}{outcolor}{12}{\boxspacing}
\begin{Verbatim}[commandchars=\\\{\}]
0    False
1    False
2    False
3    False
4    False
dtype: bool
\end{Verbatim}
\end{tcolorbox}
        
    Hint 1

Refer to the example given for how to use the \texttt{any()} method for
dataframes.

    Hint 2

Using the provided example as a guide, which axis returns \textbf{rows}
that have at least one \texttt{True} value?

    Hint 3

\texttt{mask.any(axis=1)} will return a Boolean series that represents
whether each row of \texttt{mask} contains at least one \texttt{True}
value.

    \begin{enumerate}
\def\labelenumi{\arabic{enumi}.}
\setcounter{enumi}{2}
\tightlist
\item
  Because \texttt{mask} is now a series of Boolean values, you can use
  it as a Boolean mask. Apply the Boolean mask to the
  \texttt{df\_companies} dataframe to return a filtered dataframe
  containing just the rows that contain a missing value. Assign the
  results to a variable called \texttt{df\_missing\_rows}.
\end{enumerate}

    \begin{tcolorbox}[breakable, size=fbox, boxrule=1pt, pad at break*=1mm,colback=cellbackground, colframe=cellborder]
\prompt{In}{incolor}{13}{\boxspacing}
\begin{Verbatim}[commandchars=\\\{\}]
\PY{c+c1}{\PYZsh{} 3. Apply `mask` as a Boolean mask to `df\PYZus{}companies` and assign results to `df\PYZus{}missing\PYZus{}rows`}

\PY{c+c1}{\PYZsh{}\PYZsh{}\PYZsh{} YOUR CODE HERE \PYZsh{}\PYZsh{}\PYZsh{}}
\PY{n}{df\PYZus{}missing\PYZus{}rows} \PY{o}{=} \PY{n}{df\PYZus{}companies}\PY{p}{[}\PY{n}{mask}\PY{p}{]}
\PY{n}{df\PYZus{}missing\PYZus{}rows}
\end{Verbatim}
\end{tcolorbox}

            \begin{tcolorbox}[breakable, size=fbox, boxrule=.5pt, pad at break*=1mm, opacityfill=0]
\prompt{Out}{outcolor}{13}{\boxspacing}
\begin{Verbatim}[commandchars=\\\{\}]
                         Company Valuation Date Joined  \textbackslash{}
12                           FTX      \$32B     7/20/21
170                    HyalRoute       \$4B     5/26/20
242                       Moglix       \$3B     5/17/21
251                         Trax       \$3B     7/22/19
325                  Amber Group       \$3B     6/21/21
382                    Ninja Van       \$2B     9/27/21
541   Advance Intelligence Group       \$2B     9/23/21
629             LinkSure Network       \$1B      1/1/15
811                    Carousell       \$1B     9/15/21
848                   Matrixport       \$1B      6/1/21
880                     bolttech       \$1B      7/1/21
889                        Carro       \$1B     6/14/21
893                        Cider       \$1B      9/2/21
980                         NIUM       \$1B     7/13/21
986                          ONE       \$1B     12/8/21
994                      PatSnap       \$1B     3/16/21
1061                       WeLab       \$1B     11/8/17

                                 Industry      City Country/Region  \textbackslash{}
12                                Fintech       NaN        Bahamas
170           Mobile \& telecommunications       NaN      Singapore
242       E-commerce \& direct-to-consumer       NaN      Singapore
251               Artificial intelligence       NaN      Singapore
325                               Fintech       NaN      Hong Kong
382   Supply chain, logistics, \& delivery       NaN      Singapore
541               Artificial intelligence       NaN      Singapore
629           Mobile \& telecommunications  Shanghai          China
811       E-commerce \& direct-to-consumer       NaN      Singapore
848                               Fintech       NaN      Singapore
880                               Fintech       NaN      Singapore
889       E-commerce \& direct-to-consumer       NaN      Singapore
893       E-commerce \& direct-to-consumer       NaN      Hong Kong
980                               Fintech       NaN      Singapore
986          Internet software \& services       NaN      Singapore
994          Internet software \& services       NaN      Singapore
1061                              Fintech       NaN      Hong Kong

          Continent  Year Founded Funding  \textbackslash{}
12    North America          2018     \$2B
170            Asia          2015   \$263M
242            Asia          2015   \$471M
251            Asia          2010     \$1B
325            Asia          2015   \$328M
382            Asia          2014   \$975M
541            Asia          2016   \$536M
629            Asia          2013    \$52M
811            Asia          2012   \$288M
848            Asia          2019   \$100M
880            Asia          2018   \$210M
889            Asia          2015   \$595M
893            Asia          2020   \$140M
980            Asia          2014   \$285M
986            Asia          2011   \$515M
994            Asia          2007   \$352M
1061           Asia          2013   \$871M

                                       Select Investors  Year\_Joined  \textbackslash{}
12               Sequoia Capital, Thoma Bravo, Softbank         2021
170                                           Kuang-Chi         2020
242             Jungle Ventures, Accel, Venture Highway         2021
251   Hopu Investment Management, Boyu Capital, DC T{\ldots}         2019
325   Tiger Global Management, Tiger Brokers, DCM Ve{\ldots}         2021
382   B Capital Group, Monk's Hill Ventures, Dynamic{\ldots}         2021
541         Vision Plus Capital, GSR Ventures, ZhenFund         2021
629                                                 NaN         2015
811   500 Global, Rakuten Ventures, Golden Gate Vent{\ldots}         2021
848   Dragonfly Captial, Qiming Venture Partners, DS{\ldots}         2021
880   Mundi Ventures, Doqling Capital Partners, Acti{\ldots}         2021
889   SingTel Innov8, Alpha JWC Ventures, Golden Gat{\ldots}         2021
893        Andreessen Horowitz, DST Global, IDG Capital         2021
980   Vertex Ventures SE Asia, Global Founders Capit{\ldots}         2021
986   Temasek, Guggenheim Investments, Qatar Investm{\ldots}         2021
994   Sequoia Capital China, Shunwei Capital Partner{\ldots}         2021
1061  Sequoia Capital China, ING, Alibaba Entreprene{\ldots}         2017

      valuation\_num
12               32
170               4
242               3
251               3
325               3
382               2
541               2
629               1
811               1
848               1
880               1
889               1
893               1
980               1
986               1
994               1
1061              1
\end{Verbatim}
\end{tcolorbox}
        
    \textbf{Question: Is there a specific country/region that shows up a lot
in this missing values dataframe? Which one?} - Twelve of the 17 rows
with missing values are for companies from Singapore.

    \textbf{Question: What steps did you take to find missing data?} -
\texttt{DataFrame.isna()} will return a Boolean dataframe indicating
every location that is \texttt{NaN} with \texttt{True} - You can use
\texttt{sum()} in conjunction with \texttt{isna()} to get the counts of
\texttt{NaN} values in each column. - You can use \texttt{any()} in
conjunction with \texttt{isna()} to create a Boolean mask, which can be
applied to the original dataframe to obtain just the rows with at least
one \texttt{NaN} value.

    \textbf{Question: What observations can be made about the forms and
context of missing data?} - Missing values can take different forms and
are usually context-specific. Not every missing value is labeled as
\texttt{na} or \texttt{None}, or \texttt{Null}.

    \textbf{Question: What other methods could you use to address missing
data?} - If possible, ask the business users for insight into the causes
of missing values and, if possible, get domain knowledge to
intelligently impute these values.

    \hypertarget{step-3-model-building}{%
\subsection{Step 3: Model building}\label{step-3-model-building}}

Think of the model you are building as the completed dataset, which you
will then use to inform the questions the investor has asked of you.

    \hypertarget{two-ways-to-address-missing-values}{%
\subsubsection{Two ways to address missing
values}\label{two-ways-to-address-missing-values}}

There are several ways to address missing values, which is critical in
EDA. The two primary methods are removing them and imputing other values
in their place. Choosing the proper method depends on the business
problem and the value the solution will add or take away from the
dataset.

Here, you will try both.

    To compare the the effect of different actions, first store the original
number of values in a variable. Create a variable called
\texttt{count\_total} that is an integer representing the total number
of values in \texttt{df\_companies}. For example, if the dataframe had 5
rows and 2 columns, then this number would be 10.

    \begin{tcolorbox}[breakable, size=fbox, boxrule=1pt, pad at break*=1mm,colback=cellbackground, colframe=cellborder]
\prompt{In}{incolor}{14}{\boxspacing}
\begin{Verbatim}[commandchars=\\\{\}]
\PY{c+c1}{\PYZsh{} Store the total number of values in a variable called `count\PYZus{}total`}

\PY{c+c1}{\PYZsh{}\PYZsh{}\PYZsh{} YOUR CODE HERE \PYZsh{}\PYZsh{}\PYZsh{}}

\PY{n}{count\PYZus{}total} \PY{o}{=} \PY{n}{df\PYZus{}companies}\PY{o}{.}\PY{n}{size}
\PY{n}{count\PYZus{}total}
\end{Verbatim}
\end{tcolorbox}

            \begin{tcolorbox}[breakable, size=fbox, boxrule=.5pt, pad at break*=1mm, opacityfill=0]
\prompt{Out}{outcolor}{14}{\boxspacing}
\begin{Verbatim}[commandchars=\\\{\}]
12888
\end{Verbatim}
\end{tcolorbox}
        
    Now, remove all rows containing missing values and store the total
number of remaining values in a variable called
\texttt{count\_dropna\_rows}.

    \begin{tcolorbox}[breakable, size=fbox, boxrule=1pt, pad at break*=1mm,colback=cellbackground, colframe=cellborder]
\prompt{In}{incolor}{15}{\boxspacing}
\begin{Verbatim}[commandchars=\\\{\}]
\PY{c+c1}{\PYZsh{} Drop the rows containing missing values, determine number of remaining values }

\PY{c+c1}{\PYZsh{}\PYZsh{}\PYZsh{} YOUR CODE HERE \PYZsh{}\PYZsh{}\PYZsh{}}

\PY{n}{count\PYZus{}dropna\PYZus{}rows} \PY{o}{=} \PY{n}{df\PYZus{}companies}\PY{o}{.}\PY{n}{dropna}\PY{p}{(}\PY{p}{)}\PY{o}{.}\PY{n}{size}
\PY{n}{count\PYZus{}dropna\PYZus{}rows}
\end{Verbatim}
\end{tcolorbox}

            \begin{tcolorbox}[breakable, size=fbox, boxrule=.5pt, pad at break*=1mm, opacityfill=0]
\prompt{Out}{outcolor}{15}{\boxspacing}
\begin{Verbatim}[commandchars=\\\{\}]
12684
\end{Verbatim}
\end{tcolorbox}
        
    Hint

Use the \texttt{dropna()} dataframe method to drop rows with missing
values.

    Now, remove all columns containing missing values and store the total
number of cells in a variable called \texttt{count\_dropna\_columns}.

    \begin{tcolorbox}[breakable, size=fbox, boxrule=1pt, pad at break*=1mm,colback=cellbackground, colframe=cellborder]
\prompt{In}{incolor}{16}{\boxspacing}
\begin{Verbatim}[commandchars=\\\{\}]
\PY{c+c1}{\PYZsh{} Drop the columns containing missing values, determine number of remaining values}

\PY{c+c1}{\PYZsh{}\PYZsh{}\PYZsh{} YOUR CODE HERE \PYZsh{}\PYZsh{}\PYZsh{}}

\PY{n}{count\PYZus{}dropna\PYZus{}columns} \PY{o}{=} \PY{n}{df\PYZus{}companies}\PY{o}{.}\PY{n}{dropna}\PY{p}{(}\PY{n}{axis}\PY{o}{=}\PY{l+m+mi}{1}\PY{p}{)}\PY{o}{.}\PY{n}{size}
\PY{n}{count\PYZus{}dropna\PYZus{}columns}
\end{Verbatim}
\end{tcolorbox}

            \begin{tcolorbox}[breakable, size=fbox, boxrule=.5pt, pad at break*=1mm, opacityfill=0]
\prompt{Out}{outcolor}{16}{\boxspacing}
\begin{Verbatim}[commandchars=\\\{\}]
10740
\end{Verbatim}
\end{tcolorbox}
        
    Hint

Specify \texttt{axis=1} to the \texttt{dropna()} method to drop columns
with missing values.

    Next, print the percentage of values removed by each method and compare
them.

    \begin{tcolorbox}[breakable, size=fbox, boxrule=1pt, pad at break*=1mm,colback=cellbackground, colframe=cellborder]
\prompt{In}{incolor}{17}{\boxspacing}
\begin{Verbatim}[commandchars=\\\{\}]
\PY{c+c1}{\PYZsh{} Print the percentage of values removed by dropping rows.}

\PY{c+c1}{\PYZsh{}\PYZsh{}\PYZsh{} YOUR CODE HERE \PYZsh{}\PYZsh{}\PYZsh{}}

\PY{n}{row\PYZus{}percent} \PY{o}{=} \PY{p}{(}\PY{p}{(}\PY{n}{count\PYZus{}total} \PY{o}{\PYZhy{}} \PY{n}{count\PYZus{}dropna\PYZus{}rows}\PY{p}{)} \PY{o}{/} \PY{n}{count\PYZus{}total}\PY{p}{)} \PY{o}{*} \PY{l+m+mi}{100}
\PY{n+nb}{print}\PY{p}{(}\PY{l+s+sa}{f}\PY{l+s+s1}{\PYZsq{}}\PY{l+s+s1}{Percentage removed, rows: }\PY{l+s+si}{\PYZob{}}\PY{n}{row\PYZus{}percent}\PY{l+s+si}{:}\PY{l+s+s1}{.3f}\PY{l+s+si}{\PYZcb{}}\PY{l+s+s1}{\PYZsq{}}\PY{p}{)}

\PY{c+c1}{\PYZsh{} Print the percentage of values removed by dropping columns.}

\PY{c+c1}{\PYZsh{}\PYZsh{}\PYZsh{} YOUR CODE HERE \PYZsh{}\PYZsh{}\PYZsh{}}

\PY{n}{col\PYZus{}percent} \PY{o}{=} \PY{p}{(}\PY{p}{(}\PY{n}{count\PYZus{}total} \PY{o}{\PYZhy{}} \PY{n}{count\PYZus{}dropna\PYZus{}columns}\PY{p}{)} \PY{o}{/} \PY{n}{count\PYZus{}total}\PY{p}{)} \PY{o}{*} \PY{l+m+mi}{100}
\PY{n+nb}{print}\PY{p}{(}\PY{l+s+sa}{f}\PY{l+s+s1}{\PYZsq{}}\PY{l+s+s1}{Percentage removed, columns: }\PY{l+s+si}{\PYZob{}}\PY{n}{col\PYZus{}percent}\PY{l+s+si}{:}\PY{l+s+s1}{.3f}\PY{l+s+si}{\PYZcb{}}\PY{l+s+s1}{\PYZsq{}}\PY{p}{)}
\end{Verbatim}
\end{tcolorbox}

    \begin{Verbatim}[commandchars=\\\{\}]
Percentage removed, rows: 1.583
Percentage removed, columns: 16.667
    \end{Verbatim}

    \textbf{Question: Which method was most effective? Why?}

The percentage removed was significantly higher for columns than it was
for rows. Since both approaches result in a dataset with no missing
values, the ``most effective'' method depends on how much data you have
and what you want to do with it. It might be best to use the way that
leaves the most data intact---in this case, dropping rows. Or, if you
don't have many samples and don't want to lose any, but you don't need
all your columns, then dropping columns might be best. With this data,
it would probably be best to drop rows in the majority of cases.

    Now, practice the second method: imputation. Perform the following
steps:

\begin{enumerate}
\def\labelenumi{\arabic{enumi}.}
\tightlist
\item
  Use the
  \href{https://pandas.pydata.org/docs/reference/api/pandas.DataFrame.fillna.html\#pandas.DataFrame.fillna}{\texttt{fillna()}}
  dataframe method to fill each missing value with the next non-NaN
  value in its column. Assign the results to a new dataframe called
  \texttt{df\_companies\_backfill}.
\end{enumerate}

\begin{verbatim}
Example:

df =     
        A    B    C
    0   5    a    NaN
    1   10   NaN  False
    2   NaN  c    True

 [IN]: df.fillna(method='backfill')
 
[OUT]: 
        A    B    C
    0   5    a    False
    1   10   c    False
    2   NaN  c    True
\end{verbatim}

Notice that if there is a \texttt{NaN} value in the last row, it will
not backfill because there is no subsequent value in the column to refer
to.

\begin{enumerate}
\def\labelenumi{\arabic{enumi}.}
\setcounter{enumi}{1}
\tightlist
\item
  Show the rows that previously had missing values.
\end{enumerate}

    \begin{tcolorbox}[breakable, size=fbox, boxrule=1pt, pad at break*=1mm,colback=cellbackground, colframe=cellborder]
\prompt{In}{incolor}{18}{\boxspacing}
\begin{Verbatim}[commandchars=\\\{\}]
\PY{c+c1}{\PYZsh{} 1. Fill missing values using the \PYZsq{}fillna()\PYZsq{} method, back\PYZhy{}filling}

\PY{c+c1}{\PYZsh{}\PYZsh{}\PYZsh{} YOUR CODE HERE \PYZsh{}\PYZsh{}\PYZsh{}}

\PY{n}{df\PYZus{}companies\PYZus{}backfill} \PY{o}{=} \PY{n}{df\PYZus{}companies}\PY{o}{.}\PY{n}{fillna}\PY{p}{(}\PY{n}{method}\PY{o}{=}\PY{l+s+s1}{\PYZsq{}}\PY{l+s+s1}{backfill}\PY{l+s+s1}{\PYZsq{}}\PY{p}{)}

\PY{c+c1}{\PYZsh{} 2. Show the rows that previously had missing values}

\PY{c+c1}{\PYZsh{}\PYZsh{}\PYZsh{} YOUR CODE HERE \PYZsh{}\PYZsh{}\PYZsh{}}

\PY{n}{df\PYZus{}companies\PYZus{}backfill}\PY{o}{.}\PY{n}{iloc}\PY{p}{[}\PY{n}{df\PYZus{}missing\PYZus{}rows}\PY{o}{.}\PY{n}{index}\PY{p}{,} \PY{p}{:}\PY{p}{]}
\end{Verbatim}
\end{tcolorbox}

            \begin{tcolorbox}[breakable, size=fbox, boxrule=.5pt, pad at break*=1mm, opacityfill=0]
\prompt{Out}{outcolor}{18}{\boxspacing}
\begin{Verbatim}[commandchars=\\\{\}]
                         Company Valuation Date Joined  \textbackslash{}
12                           FTX      \$32B     7/20/21
170                    HyalRoute       \$4B     5/26/20
242                       Moglix       \$3B     5/17/21
251                         Trax       \$3B     7/22/19
325                  Amber Group       \$3B     6/21/21
382                    Ninja Van       \$2B     9/27/21
541   Advance Intelligence Group       \$2B     9/23/21
629             LinkSure Network       \$1B      1/1/15
811                    Carousell       \$1B     9/15/21
848                   Matrixport       \$1B      6/1/21
880                     bolttech       \$1B      7/1/21
889                        Carro       \$1B     6/14/21
893                        Cider       \$1B      9/2/21
980                         NIUM       \$1B     7/13/21
986                          ONE       \$1B     12/8/21
994                      PatSnap       \$1B     3/16/21
1061                       WeLab       \$1B     11/8/17

                                 Industry           City Country/Region  \textbackslash{}
12                                Fintech   Jacksonville        Bahamas
170           Mobile \& telecommunications     El Segundo      Singapore
242       E-commerce \& direct-to-consumer  San Francisco      Singapore
251               Artificial intelligence      Amsterdam      Singapore
325                               Fintech  San Francisco      Hong Kong
382   Supply chain, logistics, \& delivery  San Francisco      Singapore
541               Artificial intelligence       Helsinki      Singapore
629           Mobile \& telecommunications       Shanghai          China
811       E-commerce \& direct-to-consumer       New York      Singapore
848                               Fintech  San Francisco      Singapore
880                               Fintech      Englewood      Singapore
889       E-commerce \& direct-to-consumer        Lincoln      Singapore
893       E-commerce \& direct-to-consumer    Mexico City      Hong Kong
980                               Fintech      Bengaluru      Singapore
986          Internet software \& services       New York      Singapore
994          Internet software \& services         London      Singapore
1061                              Fintech        Beijing      Hong Kong

          Continent  Year Founded Funding  \textbackslash{}
12    North America          2018     \$2B
170            Asia          2015   \$263M
242            Asia          2015   \$471M
251            Asia          2010     \$1B
325            Asia          2015   \$328M
382            Asia          2014   \$975M
541            Asia          2016   \$536M
629            Asia          2013    \$52M
811            Asia          2012   \$288M
848            Asia          2019   \$100M
880            Asia          2018   \$210M
889            Asia          2015   \$595M
893            Asia          2020   \$140M
980            Asia          2014   \$285M
986            Asia          2011   \$515M
994            Asia          2007   \$352M
1061           Asia          2013   \$871M

                                       Select Investors  Year\_Joined  \textbackslash{}
12               Sequoia Capital, Thoma Bravo, Softbank         2021
170                                           Kuang-Chi         2020
242             Jungle Ventures, Accel, Venture Highway         2021
251   Hopu Investment Management, Boyu Capital, DC T{\ldots}         2019
325   Tiger Global Management, Tiger Brokers, DCM Ve{\ldots}         2021
382   B Capital Group, Monk's Hill Ventures, Dynamic{\ldots}         2021
541         Vision Plus Capital, GSR Ventures, ZhenFund         2021
629   Sequoia Capital India, The Times Group, GMO Ve{\ldots}         2015
811   500 Global, Rakuten Ventures, Golden Gate Vent{\ldots}         2021
848   Dragonfly Captial, Qiming Venture Partners, DS{\ldots}         2021
880   Mundi Ventures, Doqling Capital Partners, Acti{\ldots}         2021
889   SingTel Innov8, Alpha JWC Ventures, Golden Gat{\ldots}         2021
893        Andreessen Horowitz, DST Global, IDG Capital         2021
980   Vertex Ventures SE Asia, Global Founders Capit{\ldots}         2021
986   Temasek, Guggenheim Investments, Qatar Investm{\ldots}         2021
994   Sequoia Capital China, Shunwei Capital Partner{\ldots}         2021
1061  Sequoia Capital China, ING, Alibaba Entreprene{\ldots}         2017

      valuation\_num
12               32
170               4
242               3
251               3
325               3
382               2
541               2
629               1
811               1
848               1
880               1
889               1
893               1
980               1
986               1
994               1
1061              1
\end{Verbatim}
\end{tcolorbox}
        
    Hint 1

To backfill missing values, refer to the example provided.

    Hint 2

To show the rows that previously had missing values, you'll need the
indices of the rows that had missing values.

    Hint 3

\begin{itemize}
\tightlist
\item
  You already have a dataframe of rows with missing values. It's stored
  in a variable called \texttt{df\_missing\_rows}.\\
\item
  To access its index, call \texttt{df\_missing\_rows.index}. This will
  give you the row numbers of rows with missing values.\\
\item
  Use these index numbers in an iloc{[}{]} selection statement on the
  \texttt{df\_companies\_backfill} dataframe to extract those row
  numbers.
\end{itemize}

    \textbf{Question: Do the values that were used to fill in for the
missing values make sense?} - No, the values seem to be added without
consideration of the country those cities are located in.

    Another option is to fill the values with a certain value, such as
`Unknown'. However, doing so doesn't add any value to the dataset and
could make finding the missing values difficult in the future. Reviewing
the missing values in this dataset determines that it is fine to leave
the values as they are. This also avoids adding bias to the dataset.

    \hypertarget{step-4-results-and-evaluation}{%
\subsection{Step 4: Results and
evaluation}\label{step-4-results-and-evaluation}}

    Now that you've addressed your missing values, provide your investor
with their requested data points.

    \hypertarget{companies-in-the-hardware-industry}{%
\subsubsection{\texorpdfstring{Companies in the \texttt{Hardware}
Industry}{Companies in the Hardware Industry}}\label{companies-in-the-hardware-industry}}

Your investor is interested in identifying unicorn companies in the
\texttt{Hardware} industry in the following cities: \texttt{Beijing},
\texttt{San\ Francisco}, and \texttt{London}. They are also interested
in companies in the \texttt{Artificial\ intelligence} industry in
\texttt{London}.

Write a selection statement that extracts the rows that meet these
criteria. This task requires complex conditional logic. Break the
process into the following parts.

\begin{enumerate}
\def\labelenumi{\arabic{enumi}.}
\tightlist
\item
  Create a mask to apply to the \texttt{df\_companies} dataframe. The
  following logic is a pseudo-code representation of how this mask could
  be structured.
\end{enumerate}

\begin{verbatim}
((Industry==Hardware) and (City==Beijing, San Francisco, or London)) 
OR  
((Industry==Artificial intelligence) and (City==London))
\end{verbatim}

You're familiar with how to create Boolean masks based on conditional
logic in pandas. However, you might not know how to write a conditional
statement that selects rows that have \emph{any one of several possible
values} in a given column. In this case, this is the
\texttt{(City==Beijing,\ San\ Francisco,\ or\ London)} part of the
expression.

For this type of construction, use the
\href{https://pandas.pydata.org/docs/reference/api/pandas.Series.isin.html\#pandas.Series.isin}{\texttt{isin()}}
\texttt{Series} method. This method is applied to a pandas series and,
for each value in the series, checks whether it is a member of whatever
is passed as its argument.

\begin{verbatim}
Example:

 [IN]: my_series = pd.Series([0, 1, 2, 3])
       my_series
       
[OUT]: 0    0
       1    1
       2    2
       3    3
       dtype: int64
       
 [IN]: my_series.isin([1, 2])
       
[OUT]: 0    False
       1     True
       2     True
       3    False
       dtype: bool
       
\end{verbatim}

\begin{enumerate}
\def\labelenumi{\arabic{enumi}.}
\setcounter{enumi}{1}
\tightlist
\item
  Apply the mask to the \texttt{df\_companies} dataframe and assign the
  result to a new variable called \texttt{df\_invest}.
\end{enumerate}

    \begin{tcolorbox}[breakable, size=fbox, boxrule=1pt, pad at break*=1mm,colback=cellbackground, colframe=cellborder]
\prompt{In}{incolor}{19}{\boxspacing}
\begin{Verbatim}[commandchars=\\\{\}]
\PY{c+c1}{\PYZsh{} 1. Create a Boolean mask using conditional logic}

\PY{c+c1}{\PYZsh{}\PYZsh{}\PYZsh{} YOUR CODE HERE \PYZsh{}\PYZsh{}\PYZsh{}}

\PY{n}{cities} \PY{o}{=} \PY{p}{[}\PY{l+s+s1}{\PYZsq{}}\PY{l+s+s1}{Beijing}\PY{l+s+s1}{\PYZsq{}}\PY{p}{,} \PY{l+s+s1}{\PYZsq{}}\PY{l+s+s1}{San Francisco}\PY{l+s+s1}{\PYZsq{}}\PY{p}{,} \PY{l+s+s1}{\PYZsq{}}\PY{l+s+s1}{London}\PY{l+s+s1}{\PYZsq{}}\PY{p}{]}
\PY{n}{mask} \PY{o}{=} \PY{p}{(}
    \PY{p}{(}\PY{n}{df\PYZus{}companies}\PY{p}{[}\PY{l+s+s1}{\PYZsq{}}\PY{l+s+s1}{Industry}\PY{l+s+s1}{\PYZsq{}}\PY{p}{]}\PY{o}{==}\PY{l+s+s1}{\PYZsq{}}\PY{l+s+s1}{Hardware}\PY{l+s+s1}{\PYZsq{}}\PY{p}{)} \PY{o}{\PYZam{}} \PY{p}{(}\PY{n}{df\PYZus{}companies}\PY{p}{[}\PY{l+s+s1}{\PYZsq{}}\PY{l+s+s1}{City}\PY{l+s+s1}{\PYZsq{}}\PY{p}{]}\PY{o}{.}\PY{n}{isin}\PY{p}{(}\PY{n}{cities}\PY{p}{)}\PY{p}{)}
\PY{p}{)} \PY{o}{|} \PY{p}{(}
    \PY{p}{(}\PY{n}{df\PYZus{}companies}\PY{p}{[}\PY{l+s+s1}{\PYZsq{}}\PY{l+s+s1}{Industry}\PY{l+s+s1}{\PYZsq{}}\PY{p}{]}\PY{o}{==}\PY{l+s+s1}{\PYZsq{}}\PY{l+s+s1}{Artificial intelligence}\PY{l+s+s1}{\PYZsq{}}\PY{p}{)} \PY{o}{\PYZam{}} \PY{p}{(}\PY{n}{df\PYZus{}companies}\PY{p}{[}\PY{l+s+s1}{\PYZsq{}}\PY{l+s+s1}{City}\PY{l+s+s1}{\PYZsq{}}\PY{p}{]}\PY{o}{==}\PY{l+s+s1}{\PYZsq{}}\PY{l+s+s1}{London}\PY{l+s+s1}{\PYZsq{}}\PY{p}{)}
\PY{p}{)}

\PY{c+c1}{\PYZsh{} 2. Apply the mask to the `df\PYZus{}companies` dataframe and assign the results to `df\PYZus{}invest`}

\PY{c+c1}{\PYZsh{}\PYZsh{}\PYZsh{} YOUR CODE HERE \PYZsh{}\PYZsh{}\PYZsh{}}
\PY{n}{df\PYZus{}invest} \PY{o}{=} \PY{n}{df\PYZus{}companies}\PY{p}{[}\PY{n}{mask}\PY{p}{]}
\PY{n}{df\PYZus{}invest}
\end{Verbatim}
\end{tcolorbox}

            \begin{tcolorbox}[breakable, size=fbox, boxrule=.5pt, pad at break*=1mm, opacityfill=0]
\prompt{Out}{outcolor}{19}{\boxspacing}
\begin{Verbatim}[commandchars=\\\{\}]
                  Company Valuation Date Joined                 Industry  \textbackslash{}
36                Bitmain      \$12B      7/6/18                 Hardware
43          Global Switch      \$11B    12/22/16                 Hardware
147               Chipone       \$5B    12/16/21                 Hardware
845               Density       \$1B    11/10/21                 Hardware
873          BenevolentAI       \$1B      6/2/15  Artificial intelligence
923                 Geek+       \$1B    11/21/18                 Hardware
1040  TERMINUS Technology       \$1B    10/25/18                 Hardware
1046            Tractable       \$1B     6/16/21  Artificial intelligence

               City  Country/Region      Continent  Year Founded Funding  \textbackslash{}
36          Beijing           China           Asia          2015   \$765M
43           London  United Kingdom         Europe          1998     \$5B
147         Beijing           China           Asia          2008     \$1B
845   San Francisco   United States  North America          2014   \$217M
873          London  United Kingdom         Europe          2013   \$292M
923         Beijing           China           Asia          2015   \$439M
1040        Beijing           China           Asia          2015   \$623M
1046         London  United Kingdom         Europe          2014   \$120M

                                       Select Investors  Year\_Joined  \textbackslash{}
36    Coatue Management, Sequoia Capital China, IDG {\ldots}         2018
43    Aviation Industry Corporation of China, Essenc{\ldots}         2016
147   China Grand Prosperity Investment, Silk Road H{\ldots}         2021
845        Founders Fund, Upfront Ventures, 01 Advisors         2021
873                      Woodford Investment Management         2015
923   Volcanics Ventures, Vertex Ventures China, War{\ldots}         2018
1040     China Everbright Limited, IDG Capital, iFLYTEK         2018
1046  Insight Partners, Ignition Partners, Georgian {\ldots}         2021

      valuation\_num
36               12
43               11
147               5
845               1
873               1
923               1
1040              1
1046              1
\end{Verbatim}
\end{tcolorbox}
        
    Hint 1

\begin{itemize}
\tightlist
\item
  Remember that pandas uses \texttt{\&} for ``and'', \texttt{\textbar{}}
  for ``or'', and \texttt{\textasciitilde{}} for ``not''.

  \begin{itemize}
  \tightlist
  \item
    Remember that each condition needs to be in its own set of
    parentheses. Refer to the above pseudo-code for an example.
  \end{itemize}
\end{itemize}

    Hint 2

\begin{itemize}
\tightlist
\item
  Use \texttt{(Series.isin(list\_of\_cities))} to represent the logic:
  (City==Beijing, San Francisco, or London)`.

  \begin{itemize}
  \tightlist
  \item
    There are two sets of conditional pairs: ((A) and (B)) or ((C) and
    (D)). Make sure the parentheses reflect this logic.
  \end{itemize}
\end{itemize}

    Hint 3

Consider using the following code:

\texttt{cities\ =\ {[}\textquotesingle{}Beijing\textquotesingle{},\ \textquotesingle{}San\ Francisco\textquotesingle{},\ \textquotesingle{}London\textquotesingle{}{]}\ \ \ \ \ mask\ =\ (\ \ \ \ \ \ \ \ \ (df\_companies{[}\textquotesingle{}Industry\textquotesingle{}{]}==\textquotesingle{}Hardware\textquotesingle{})\ \&\ (df\_companies{[}\textquotesingle{}City\textquotesingle{}{]}.isin(cities))\ \ \ \ \ )\ \textbar{}\ (\ \ \ \ \ \ \ \ \ (df\_companies{[}\textquotesingle{}Industry\textquotesingle{}{]}==\textquotesingle{}Artificial\ intelligence\textquotesingle{})\ \&\ (df\_companies{[}\textquotesingle{}City\textquotesingle{}{]}==\textquotesingle{}London\textquotesingle{})\ \ \ \ \ )\ \ \ \ \ df\_invest\ =\ df\_companies{[}mask{]}}

    \textbf{Question: How many companies meet the criteria given by the
investor?} - Eight companies meet the stated criteria.

    \hypertarget{list-of-countries-by-sum-of-valuation}{%
\subsubsection{List of countries by sum of
valuation}\label{list-of-countries-by-sum-of-valuation}}

    For each country, sum the valuations of all companies in that country,
then sort the results in descending order by summed valuation. Assign
the results to a variable called \texttt{national\_valuations}.

    \begin{tcolorbox}[breakable, size=fbox, boxrule=1pt, pad at break*=1mm,colback=cellbackground, colframe=cellborder]
\prompt{In}{incolor}{20}{\boxspacing}
\begin{Verbatim}[commandchars=\\\{\}]
\PY{c+c1}{\PYZsh{} Group the data by`Country/Region`}

\PY{c+c1}{\PYZsh{}\PYZsh{}\PYZsh{} YOUR CODE HERE \PYZsh{}\PYZsh{}\PYZsh{}}

\PY{n}{national\PYZus{}valuations} \PY{o}{=} \PY{n}{df\PYZus{}companies}\PY{o}{.}\PY{n}{groupby}\PY{p}{(}\PY{p}{[}\PY{l+s+s1}{\PYZsq{}}\PY{l+s+s1}{Country/Region}\PY{l+s+s1}{\PYZsq{}}\PY{p}{]}\PY{p}{)}\PY{p}{[}\PY{l+s+s1}{\PYZsq{}}\PY{l+s+s1}{valuation\PYZus{}num}\PY{l+s+s1}{\PYZsq{}}\PY{p}{]}\PY{o}{.}\PY{n}{sum}\PY{p}{(}
\PY{p}{)}\PY{o}{.}\PY{n}{sort\PYZus{}values}\PY{p}{(}\PY{n}{ascending}\PY{o}{=}\PY{k+kc}{False}\PY{p}{)}\PY{o}{.}\PY{n}{reset\PYZus{}index}\PY{p}{(}\PY{p}{)}

\PY{c+c1}{\PYZsh{} Print the top 15 values of the DataFrame.}

\PY{c+c1}{\PYZsh{}\PYZsh{}\PYZsh{} YOUR CODE HERE \PYZsh{}\PYZsh{}\PYZsh{}}

\PY{n}{national\PYZus{}valuations}\PY{o}{.}\PY{n}{head}\PY{p}{(}\PY{l+m+mi}{15}\PY{p}{)}
\end{Verbatim}
\end{tcolorbox}

            \begin{tcolorbox}[breakable, size=fbox, boxrule=.5pt, pad at break*=1mm, opacityfill=0]
\prompt{Out}{outcolor}{20}{\boxspacing}
\begin{Verbatim}[commandchars=\\\{\}]
    Country/Region  valuation\_num
0    United States           1933
1            China            696
2            India            196
3   United Kingdom            195
4          Germany             72
5           Sweden             63
6        Australia             56
7           France             55
8           Canada             49
9      South Korea             41
10          Israel             39
11          Brazil             37
12         Bahamas             32
13       Indonesia             28
14       Singapore             21
\end{Verbatim}
\end{tcolorbox}
        
    Hint

Use a \texttt{groupby()} statement to group by \texttt{Country/Region},
then isolate the \texttt{valuation\_num} column, sum it, and use the
\texttt{sort\_values()} method to sort the results.

    \textbf{Question: Which countries have the highest sum of valuation?}

The sorted data indicates that the four countries with highest total
company valuations are the United States, China, India, and the United
Kingdom. However, your investor specified that these countries should
not be included in the list because they are outliers.

    \hypertarget{filter-out-top-4-outlying-countries}{%
\subsubsection{Filter out top 4 outlying
countries}\label{filter-out-top-4-outlying-countries}}

    Use this grouped and summed data to plot a barplot. However, to meet the
needs of your stakeholder, you must first remove the United States,
China, India, and the United Kingdom. Remove these countries from
\texttt{national\_valuations} and reassign the results to a variable
called \texttt{national\_valuations\_no\_big4}.

    \begin{tcolorbox}[breakable, size=fbox, boxrule=1pt, pad at break*=1mm,colback=cellbackground, colframe=cellborder]
\prompt{In}{incolor}{21}{\boxspacing}
\begin{Verbatim}[commandchars=\\\{\}]
\PY{c+c1}{\PYZsh{} Remove outlying countries}

\PY{c+c1}{\PYZsh{}\PYZsh{}\PYZsh{} YOUR CODE HERE \PYZsh{}\PYZsh{}\PYZsh{}}

\PY{n}{national\PYZus{}valuations\PYZus{}no\PYZus{}big4} \PY{o}{=} \PY{n}{national\PYZus{}valuations}\PY{o}{.}\PY{n}{iloc}\PY{p}{[}\PY{l+m+mi}{4}\PY{p}{:}\PY{p}{,} \PY{p}{:}\PY{p}{]}

\PY{n}{national\PYZus{}valuations\PYZus{}no\PYZus{}big4}\PY{o}{.}\PY{n}{head}\PY{p}{(}\PY{p}{)}
\end{Verbatim}
\end{tcolorbox}

            \begin{tcolorbox}[breakable, size=fbox, boxrule=.5pt, pad at break*=1mm, opacityfill=0]
\prompt{Out}{outcolor}{21}{\boxspacing}
\begin{Verbatim}[commandchars=\\\{\}]
  Country/Region  valuation\_num
4        Germany             72
5         Sweden             63
6      Australia             56
7         France             55
8         Canada             49
\end{Verbatim}
\end{tcolorbox}
        
    Hint

There are a number of ways to accomplish this task. One of the easiest
ways is to use a simple iloc{[}{]} selection statement to select row
indices 4--end and all columns of \texttt{national\_valuations}.

    \hypertarget{bonus-content-alternative-approach-optional}{%
\subsubsection{BONUS CONTENT: Alternative approach
(optional)}\label{bonus-content-alternative-approach-optional}}

You can also use \texttt{isin()} to create a Boolean mask to filter out
specific values of the \texttt{Country/Region} column. In this case,
this process is longer and more complicated than simply using the
iloc{[}{]} statement. However, there will be situations where this is
the most direct approach.

How could you use \texttt{isin()} and your knowledge of pandas
conditional operators and Boolean masks to accomplish the same task?

    \begin{tcolorbox}[breakable, size=fbox, boxrule=1pt, pad at break*=1mm,colback=cellbackground, colframe=cellborder]
\prompt{In}{incolor}{22}{\boxspacing}
\begin{Verbatim}[commandchars=\\\{\}]
\PY{c+c1}{\PYZsh{} (Optional) Use `isin()` to create a Boolean mask to accomplish the same task}

\PY{c+c1}{\PYZsh{}\PYZsh{}\PYZsh{} YOUR CODE HERE \PYZsh{}\PYZsh{}\PYZsh{}}

\PY{n}{mask} \PY{o}{=} \PY{o}{\PYZti{}}\PY{n}{national\PYZus{}valuations}\PY{p}{[}\PY{l+s+s1}{\PYZsq{}}\PY{l+s+s1}{Country/Region}\PY{l+s+s1}{\PYZsq{}}\PY{p}{]}\PY{o}{.}\PY{n}{isin}\PY{p}{(}\PY{p}{[}\PY{l+s+s1}{\PYZsq{}}\PY{l+s+s1}{United States}\PY{l+s+s1}{\PYZsq{}}\PY{p}{,} \PY{l+s+s1}{\PYZsq{}}\PY{l+s+s1}{China}\PY{l+s+s1}{\PYZsq{}}\PY{p}{,} \PY{l+s+s1}{\PYZsq{}}\PY{l+s+s1}{India}\PY{l+s+s1}{\PYZsq{}}\PY{p}{,} \PY{l+s+s1}{\PYZsq{}}\PY{l+s+s1}{United Kingdom}\PY{l+s+s1}{\PYZsq{}}\PY{p}{]}\PY{p}{)}
\PY{n}{national\PYZus{}valuations\PYZus{}no\PYZus{}big4} \PY{o}{=} \PY{n}{national\PYZus{}valuations}\PY{p}{[}\PY{n}{mask}\PY{p}{]}
\PY{n}{national\PYZus{}valuations\PYZus{}no\PYZus{}big4}\PY{o}{.}\PY{n}{head}\PY{p}{(}\PY{p}{)}
\end{Verbatim}
\end{tcolorbox}

            \begin{tcolorbox}[breakable, size=fbox, boxrule=.5pt, pad at break*=1mm, opacityfill=0]
\prompt{Out}{outcolor}{22}{\boxspacing}
\begin{Verbatim}[commandchars=\\\{\}]
  Country/Region  valuation\_num
4        Germany             72
5         Sweden             63
6      Australia             56
7         France             55
8         Canada             49
\end{Verbatim}
\end{tcolorbox}
        
    Answer

In this case, there are 46 total countries and you want to keep
countries 5--46 and filter out countries 1--4. To use \texttt{isin()}
would require you to list out 42 countries:

\begin{verbatim}
mask = national_valuations['Country/Region'].isin(['country_5', 'country_6', ... 'country_46'])
\end{verbatim}

This is very impractical. However, you can invert the statement to
simplify the job. The above impractical statement becomes:

\begin{verbatim}
mask = ~national_valuations['Country/Region'].isin(['country_1', 'country_2', 'country_3', 'country_4'])
\end{verbatim}

Notice the \texttt{\textasciitilde{}} that precedes the whole statement.
This transforms the meaning from ``country is in this list'' to
``country is NOT in this list.''

Then, simply apply the mask to \texttt{national\_valuations} and assign
the result back to \texttt{national\_valuations\_no\_big4}.

    \hypertarget{create-barplot-for-top-20-non-big-4-countries}{%
\subsubsection{Create barplot for top 20 non-big-4
countries}\label{create-barplot-for-top-20-non-big-4-countries}}

    Now, the data is ready to reveal the top 20 non-big-4 countries with the
highest total company valuations. Use seaborn's
\href{https://seaborn.pydata.org/generated/seaborn.barplot.html}{\texttt{barplot()}}
function to create a plot showing national valuation on one axis and
country on the other.

    \begin{tcolorbox}[breakable, size=fbox, boxrule=1pt, pad at break*=1mm,colback=cellbackground, colframe=cellborder]
\prompt{In}{incolor}{23}{\boxspacing}
\begin{Verbatim}[commandchars=\\\{\}]
\PY{c+c1}{\PYZsh{} Create a barplot to compare the top 20 non\PYZhy{}big\PYZhy{}4 countries with highest company valuations}

\PY{c+c1}{\PYZsh{}\PYZsh{}\PYZsh{} YOUR CODE HERE \PYZsh{}\PYZsh{}\PYZsh{}}

\PY{n}{sns}\PY{o}{.}\PY{n}{barplot}\PY{p}{(}\PY{n}{data}\PY{o}{=}\PY{n}{national\PYZus{}valuations\PYZus{}no\PYZus{}big4}\PY{o}{.}\PY{n}{head}\PY{p}{(}\PY{l+m+mi}{20}\PY{p}{)}\PY{p}{,}
            \PY{n}{y}\PY{o}{=}\PY{l+s+s1}{\PYZsq{}}\PY{l+s+s1}{Country/Region}\PY{l+s+s1}{\PYZsq{}}\PY{p}{,}
            \PY{n}{x}\PY{o}{=}\PY{l+s+s1}{\PYZsq{}}\PY{l+s+s1}{valuation\PYZus{}num}\PY{l+s+s1}{\PYZsq{}}\PY{p}{)}
\PY{n}{plt}\PY{o}{.}\PY{n}{title}\PY{p}{(}\PY{l+s+s1}{\PYZsq{}}\PY{l+s+s1}{Top 20 non\PYZhy{}big\PYZhy{}4 countries by total company valuation}\PY{l+s+s1}{\PYZsq{}}\PY{p}{)}
\PY{n}{plt}\PY{o}{.}\PY{n}{show}\PY{p}{(}\PY{p}{)}\PY{p}{;}
\end{Verbatim}
\end{tcolorbox}

    \begin{center}
    \adjustimage{max size={0.9\linewidth}{0.9\paperheight}}{output_105_0.png}
    \end{center}
    { \hspace*{\fill} \\}
    
    Hint 1

Select the top 20 rows in \texttt{national\_valuations\_no\_big4}.

    Hint 2

\begin{itemize}
\tightlist
\item
  Select the top 20 rows in
  \texttt{df\_companies\_sum\_outliers\_removed} by using the
  \texttt{head(20)} method.
\item
  Specify \texttt{Country/Region} for the \texttt{x} parameter of the
  function and \texttt{valuation\_num} for the \texttt{y} parameter of
  the function (or vice versa).
\end{itemize}

    \hypertarget{plot-maps}{%
\subsubsection{Plot maps}\label{plot-maps}}

    Your investor has also asked for a global valuation map of all countries
except \texttt{United\ States}, \texttt{China}, \texttt{India}, and
\texttt{United\ Kingdom} (a.k.a. ``big-four countries'').

You have learned about using
\href{https://plotly.com/python-api-reference/generated/plotly.express.scatter_geo}{\texttt{scatter\_geo()}}
from the \texttt{plotly.express} library to create plot data on a map.
Create a \texttt{scatter\_geo()} plot that depicts the total valuations
of each non-big-four country on a world map, where each valuation is
shown as a circle on the map, and the size of the circle is proportional
to that country's summed valuation.

\textbf{NOTE:} The output of the following code is a dynamic plot that
requires you to run the code to display it. To do this, go to the Cell
menu at the top of the page and select Run All.

    \begin{tcolorbox}[breakable, size=fbox, boxrule=1pt, pad at break*=1mm,colback=cellbackground, colframe=cellborder]
\prompt{In}{incolor}{24}{\boxspacing}
\begin{Verbatim}[commandchars=\\\{\}]
\PY{c+c1}{\PYZsh{} Plot the sum of valuations per country.}

\PY{n}{data} \PY{o}{=} \PY{n}{national\PYZus{}valuations\PYZus{}no\PYZus{}big4}

\PY{n}{px}\PY{o}{.}\PY{n}{scatter\PYZus{}geo}\PY{p}{(}\PY{n}{data}\PY{p}{,} 
               \PY{n}{locations}\PY{o}{=}\PY{l+s+s1}{\PYZsq{}}\PY{l+s+s1}{Country/Region}\PY{l+s+s1}{\PYZsq{}}\PY{p}{,} 
               \PY{n}{size}\PY{o}{=}\PY{l+s+s1}{\PYZsq{}}\PY{l+s+s1}{valuation\PYZus{}num}\PY{l+s+s1}{\PYZsq{}}\PY{p}{,} 
               \PY{n}{locationmode}\PY{o}{=}\PY{l+s+s1}{\PYZsq{}}\PY{l+s+s1}{country names}\PY{l+s+s1}{\PYZsq{}}\PY{p}{,} 
               \PY{n}{color}\PY{o}{=}\PY{l+s+s1}{\PYZsq{}}\PY{l+s+s1}{Country/Region}\PY{l+s+s1}{\PYZsq{}}\PY{p}{,}
               \PY{n}{title}\PY{o}{=}\PY{l+s+s1}{\PYZsq{}}\PY{l+s+s1}{Total company valuations by country (non\PYZhy{}big\PYZhy{}four)}\PY{l+s+s1}{\PYZsq{}}\PY{p}{)}
\end{Verbatim}
\end{tcolorbox}

    
    
    
    
    Hint 1

Use the \texttt{national\_valuations\_no\_big4} dataframe that you
already created.

    Hint 2

To plot the data: * Use \texttt{national\_valuations\_no\_big4} as the
\texttt{data\_frame} argument of the \texttt{scatter\_geo()} function. *
Use \texttt{\textquotesingle{}Country/Region\textquotesingle{}} as the
\texttt{locations} argument. * Use
\texttt{\textquotesingle{}country\ names\textquotesingle{}} as the
\texttt{locationmode} argument. * Use
\texttt{\textquotesingle{}Country/Region\textquotesingle{}} as the
\texttt{color} argument.

Don't forget to include a title!

    \textbf{Question: How is the valuation sum per country visualized in the
plot?} - Valuation sum per country is visualized by the size of circles
around the map.

    \textbf{Question: Does any region stand out as having a lot of
activity?} - Europe has a lot of unicorn companies in a concentrated
area.

    \hypertarget{conclusion}{%
\subsection{Conclusion}\label{conclusion}}

\textbf{What are some key takeaways that you learned during this lab?} *
Missing data is a common problem for data professionals anytime they
work with a data sample. * Addressing missing values is a part of the
data-cleaning process and an important step in EDA. * Address missing
values by either removing them or filling them in. * When considering
how to address missing values, keep in mind the business, the data, and
the questions to be answered. Always ensure you are not introducing bias
into the dataset. * Addressing the missing values enabled you to answer
your investor's questions.

\textbf{How would you present your findings from this lab to others?
Consider the information you would provide (and what you would omit),
how you would share the various data insights, and how data
visualizations could help your presentation.} * For the industry
specific companies in certain locations, you could provide a short list
of company names and locations. * For the top 20 countries by sum of
valuations, you could use the plot you created in this lab or share a
list. - For the top 20 countries sorted by sum of company valuations in
each country, you would exclude \texttt{United\ States}, \texttt{China},
\texttt{India}, and \texttt{United\ Kingdom}. * For the questions
concerning the valuation map, in addition to your visuals, you would
provide a short summary of the data points. This is because the investor
did not request a further breakdown of this data.

    \textbf{Reference}

\href{https://www.kaggle.com/datasets/mysarahmadbhat/unicorn-companies}{Bhat,
M.A.~\emph{Unicorn Companies}}

    \textbf{Congratulations!} You've completed this lab. However, you may
not notice a green check mark next to this item on Coursera's platform.
Please continue your progress regardless of the check mark. Just click
on the ``save'' icon at the top of this notebook to ensure your work has
been logged.


    % Add a bibliography block to the postdoc
    
    
    
\end{document}
